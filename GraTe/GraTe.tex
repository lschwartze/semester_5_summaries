\documentclass[a4paper, 12pt]{article}

\usepackage{fullpage}
\usepackage[utf8]{inputenc}
\usepackage[english]{babel}
\usepackage{amsmath,amssymb}
\usepackage[explicit]{titlesec}
\usepackage{ulem}
\usepackage[onehalfspacing]{setspace}

\titleformat{\subsection}
{\small}{\thesubsection}{1em}{\uline{#1}}
\begin{document}
	\begin{titlepage} 
		\title{GraTe summary}
		\clearpage\maketitle
		\thispagestyle{empty}
	\end{titlepage}
	\tableofcontents
	\section{Introduction}
	\subsection{Graphs}
	\textbf{Definition} (graph)\\
	$G = (V,E)$, as usual. Other stuff that's well known.\\
	\textbf{Definition} (graph isomorphism problems)\\
	Input: two graphs $G,H$.\\
	Output: is $G \cong H$?\\
	\textbf{Definition} (subgraph)\\
	$H \subseteq G$ is called \underline{induced} if \[\forall uv \in E(G) \text{ with } u,v \in V(H) \Rightarrow uv \in E(H)\]
	\textbf{Conjecture}\\
	Let $G_1, G_2$ be two finite graphs on at least three vertices. \[\phi: V(G_1) \to V(G_2)\] is a bijection such that \[G_1-v \cong G_2-v \; \forall v\in V(G)\]
	Then $G_1 \cong G_2$.\\
	\textbf{Definition} (unfriendly partition conjecture)\\
	A partition $(A,B)$ of the vertex set $V(G)$ is \underline{unfriendly} if $\forall a \in A:$\[\left|N(a) \cap A \right| \leq \left|N(a) \cap B\right|\] and $\forall b \in B$: \[\left|N(b) \cap B\right| \leq \left|N(b) \cap A\right|\]
	\textbf{Proposition}\\
	Every finite graph has an unfriendly partition.\\
	\textbf{Definition} (cycle double cover conjecture)\\
	Every graph with only edges that lie in cycles has a cycle double cover, i.e. a family of cycles on the graph s.t. each edge lies on exactly two cycles. Cycles can be repeated.\\
	\textbf{Theorem}\\
	Every complete graph has a cycle double cover.\\
	\textbf{Definition} (degree of a vertex)\\
	The degree of a node gives the number of edges the node lies on \[d_G(v) = \left|\{uv \in E(G)\}\right|\]
	Minimum degree of a graph $G$ \[\delta(G) = \min_{v \in V(G)} d(v)\]
	The maximum degree $\Delta(G)$ and the average degree $d(G)$ are defined similarly.\\
	The average degree can also e written as \[d(G) = \frac{1}{\left|V(G)\right|} \sum_{v \in V(G)} d_G(v) = \frac{2\left|E(G)\right|}{\left|V(G)\right|}\]
	\textbf{Lemma} (handshake lemma)\\
	The number of vertices of odd degree is always even.\\
	\textbf{Theorem} (Mantel's theorem)\\
	Every graph on $n$ vertices and more than $\frac{n^2}{4}$ edges has a triangle.\\
	\textbf{Definition}\\
	Mantel's theorem creates graphs without any triangles. For any vertex in such a graph the neighborhood doesn't have any edges. A set $X \subseteq V(G)$ with this property is called an \underline{independent set}. \\
	\section{Paths}
	\textbf{Proposition}\\
	Let $G$ be a graph. Then $G$ has a path of length $\delta(G)$. If $\delta(G) \geq 2$, then there is a cycle of length $\geq \delta(G)+1$.\\
	
	Now, let's define a relation on the vertices: \[u\sim v :\Leftrightarrow \exists \text{ a $u-v$ path}\]
	\textbf{Lemma}\\
	$\sim$ is an equivalence relation.
	\begin{itemize}
		\item symmetry: if $u-v$ is a path, $v-u$ is a path as well.
		\item transitivity: if $u-v$ and $v-w$ are paths, $u-w$ is a path.
		\item reflexivity: $u-u$ is a path.
	\end{itemize}
	\textbf{Definition}\\
	$G$ is connected, if $\forall u,v \in V(G)$ there is a $u-v$ path.\\
	The components of $G$ are induced subgraphs on the equivalence classes of $\sim$.\\
	\textbf{Definition} (Edge Density):
	\[\varepsilon(G) = \frac{\left|E(G)\right|}{\left|V(G)\right|}\]
	\textbf{Proposition}\\
	Let $G$ be a graph with at least one edge. Then $\exists H \subseteq G$ with \[\delta(H) > \varepsilon(H) \geq \varepsilon(G)\]
	\textbf{Definition}\\
	A graph $G$ is $k$-connected if there is no set (separator) $X\subseteq V(G)$ with $\left|V\right|<k$ s.t. $G-X$ is not connected. To avoid problems (e.g. with complete graphs), $\left|V(G)\right|\geq k+1$ is necessary.
	\[K(G) = \max_k \text{ s.t. $G$ is $k$-connected}\]
	\textbf{Theorem} (Mader)\\
	Let $G$ be a graph with $\varepsilon(G) \geq 2k$. Then there is $H \subseteq G$ s.t. \begin{itemize}
		\item $H$ is $(k+1)$-connected.
		\item $\varepsilon(H) > \varepsilon(G)-k$
	\end{itemize} 
	It seems that $d(G) \leq 3k$ is the right boundary.\\
	
	$X\subseteq V(G)$ is \underline{independent} if $\not\exists u,v \in X$ s.t. $(u,v) \in E(G)$. We aim to construct a graph with $d(G) \approx 3k$ that has no $(k+1)$-connected subgraph. Let $X$ be a set of $k$ independent vertices. Construct $l$ complete graphs $K_k$ (which act as cliques) and connect any vertex $v$ in $X$ with all vertices of each $K_k$. Then we notice the following: \begin{itemize}
		\item $\left|V(G)\right| = (l+1)k$
		\item $\left|E(G)\right| = lk^2 + l\begin{pmatrix}
			k\\2
		\end{pmatrix}$
	\end{itemize}
	Therefore \[d(G) = \frac{2\left|E(G)\right|}{\left|V(G)\right|} = \frac{l}{l+1}(3k-1) \; \to 3k \text{ for } l \to \infty\]
	Suppose $\exists H \subseteq G$ that is $(k+1)$-connected. \begin{itemize}
		\item $H$ may intersect at mist one of the cliques
		\item $H$ has at most one vertex in $X$
		\item $\left|V(H)\right| \leq k+1$
	\end{itemize}
	\subsection{Trees}
	\textbf{Definition}\\
	A connected Graph $T$ with no cycles is a tree.\\
	\textbf{Proposition}\\
	Every tree with $\left|V(T)\right| \geq 2$ has at least 2 leaves.\\
	\textbf{Theorem}\\
	Let $T$ be a graph. The following statements are equivalent:\begin{itemize}
		\item $T$ is a tree
		\item $T$ is \underline{minimally connected} ($T-e$ is disconnected for any $e$)
		\item $T$ is \underline{maximally acyclic} ($T+e$ has a cycle for any $e \notin E(T)$)
		\item Any $u,v \in V(T)$ have a unique path between them
	\end{itemize}
	\textbf{Definition} (Spanning tree)\\
	$T$ is a spanning tree for a graph $G$ if $T\subseteq G$ and $V(T) = V(G)$.\\
	\textbf{Theorem}\\
	Every connected graph has a spanning tree.\\
	\textbf{Theorem}
	A tree on $n$ vertices has exactly $n-1$ edges and any connected graph with $n$ vertices and $n-1$ edges is a tree.
	\subsection{Bipartite Graphs}
	\textbf{Definition}\\
	A graph $G$ is bipartite if $\exists$ partition $(A,B)$ of $V(G)$ s.t. every edge of $G$ has one endvertex in $A$ and one in $B$.\\
	If $G$ is bipartite, then every subgraph $H$ is bipartite. An example are perfect bipartite graphs.
	\subsection{Eulerpaths}
	\textbf{Definition} (walks)\\
	A walks is simply a sequence of nodes that are connected by edges. A \underline{simple} walk doesn't repeat any edge. An Euler tour of a graph $G$ is a simple, closed walk $w$ with $E(w) = E(G)$.\\
	\textbf{Theorem}\\
	A graph $G$ has an Euler tour iff $G$ is connected and $d(v)$ is even $\forall v \in V(G)$.
	\section{Matchings}
	\subsection{Foundations}
	\textbf{Definition}\\
	A \underline{matching} is an edge set $M \subseteq E(G)$ s.t. no vertex is incident with two or more edges in $M$. A \underline{maximum} matching is a matching of largest cardinality.\\
	
	Given a matching $M$, an \underline{augmenting path} $P$ is a path that \begin{itemize}
		\item starts and ends in an unmatched vertex
		\item alternates between $E(P)\setminus M$ and $M$
	\end{itemize}
	\textbf{Lemma}\\
	Given matching $M$. Then $\exists$ augmenting path $\Rightarrow$ $M$ is not maximum.\\
	
	A path $P$ is \underline{alternating} if it starts in an unmatched edge and alternates between $E(G)\setminus M$ and $M$.\\
	\textbf{Theorem} (König)\\
	Let $G$ be a bipartite graph. Then the cardinality of a maximum matching is equal to the smallest cardinality of the vertex cover.\\
	
	Let $A\cup B = V(G)$ be a partition. If all vertices in $A$ are incident with a matching edge, we have a matching of $A$. But when can we find a matching of $A$? \\
	\textbf{Theorem} (Hall)\\
	There is a matching of $A$ iff $\forall S \subseteq A$ it holds $\left|N(S)\right| \geq \left|S\right|$.\\
	\textbf{Corollary}\\
	A graph where every vertex has degree $k$ is called $k$-regular. Every $k$-regular graph has a perfect matching.\\
	\textbf{Definition}\\
	A subgraph $H$ of $G$ is a $k$-factor if \begin{enumerate}
		\item $V(H) = V(G)$
		\item $H$ is $k$-regular 
	\end{enumerate} 
	\textbf{Corollary}\\
	Every $2k$-regular graph has a $2$-factor.\\
	\textbf{Remark}\\
	Define $q(G)$ as the number of odd components of $G$. We get a new necessary condition for a perfect matching: \[\forall S \subseteq V(G) \text{ it holds } q(G-S) \leq \left|S\right|\]
	\textbf{Theorem} (Tuttle)\\
	$G$ has a perfect matching iff above remark holds.\\
	\textbf{Definition}\\
	$G$ is factor critical if $\forall v \in V(G)$ it holds, that $G-v$ has a perfect matching.\\
	\textbf{Theorem}\\
	For every $G$ there is a $S^* \subseteq V(G)$ s.t. \begin{itemize}
		\item all even components of $G-S^*$ have a perfect matching
		\item all odds components of $G-S^*$ are factor-critical
		\item $\forall \emptyset \neq X \subseteq S^*$ the number of odd components of $G-S^*$ that intersect $N(X)$ is larger than $\left|X\right|$
	\end{itemize}
	\textbf{Lemma}\\
	For every $G$ there is a $S \subseteq V(G)$ and a maximum matching $M$ s.t. $M$ covers all but at most $q(G-S) - \left|S\right|$.
	\subsection{Cubic Graphs and perfect matchings}
	\textbf{Definition}\\
	A graph is $k$-edge-connected if $\forall F \subseteq E(G)$ with $\left|F\right| \leq k-1$ it holds that $G-F$ is connected. $k$-connectivity implies $k$-edge-connectivity.\\
	\textbf{Corollary}\\
	Every $2$-edge-connected cubic graph has a perfect matching.
	\section{Connectivity}
	\subsection{How do $2$-connected graphs look like?}
	\textbf{Definition}\\
	Let $G$ be a graph. $A \subseteq V(G)$ is called an $A$-path in $G$ if $A$ is a path and only the start and end vertices of $A$ lie in $V(G)$. An ear decomposition is a sequence $G_0 \subsetneqq G_1 \subsetneqq ... \subsetneqq G_k = G$ s.t. \begin{itemize}
		\item $G_0$ is a cycle
		\item $G_{i+1}$ arises form $G_i$ by adding a $G_i$-path to $G_i$.
	\end{itemize}  
	\textbf{Theorem}\\
	A graph $G$ is 2-connected iff it has an ear-decomposition. Moreover, each $G_i$ is 2-connected.\\
	\textbf{Definition}\\
	A \underline{cut vertex} of a graph $G$ is a vertex $v \in V(G)$ s.t. $G-v$ has more components than $G$. A \underline{block} of $G$ is a maximum connected subgraph $B$ of $G$ that doesn't contain a cut vertex of $B$.\\
	\textbf{Lemma}\\
	Let $G$ be a graph \begin{enumerate}
		\item every cycle of $G$ is contained in a single block
		\item two blocks intersect only in a single vertex, that is a cut vertex of $G$
		\item $\overset{.}{\bigcup}_{B \text{ block }} E(B) = E(G)$.
	\end{enumerate}
	\textbf{Definition} (block graph)\\
	The block graph of a graph $G$ is defined as \[V(G') = \{x, B\; |\; x \text{ cut vertex in $G$ and } B \text{ block in $G$}\}, \; E(G') = \{xB: xB \in E(G)\}\].\\
	\textbf{Proposition}\\
	The block graph of a connected graph is a  tree.
	\subsection{3-connected graphs}
	\textbf{Definition}\\
	Let $G$ be a graph with an edge $xy$. We define the process of contracting an edge by the graph $G' = G/xy$ with \begin{itemize}
		\item $V(G') = (V(G)\setminus \{x,y\})\cup \{v_{xy}\}$
		\item $E(G') = \{uv \; | \; u,v \notin \{x,y\}\} \cup \{uv_{xy}: \; ux \in E(G) \lor uy \in E(G), \; u \notin \{x,y\}\}$
	\end{itemize}
	\textbf{Lemma}\\
	Let $G$ be a $3$-connected graph on at least 5 vertices. Than there is an edge $e$ s.t. $G/e$ is a $3$-connected graph.\\
	\textbf{Theorem}\\
	A graph $G$ is $3$-connected iff there is a sequence \[K_4 = G_0, G_1,...,G_k = G\] 
	where $x_iy_i \in E(G_i)$ s.t. $G_{i-1} = G_i/x_iy_i$ and $d_{G_i}(x_i) \geq 3, \; d_{G_i}(y_i) \geq 3$.\\
	\textbf{Definition}\\
	Let $A,B$ be vertex sets. $X \subseteq V(G)$ separates $A$ from $B$ if there is no $A-B$-path in $G-X$.\\
	\textbf{Theorem} (Menger)\\
	For every $A,B \subseteq V(G)$, $k \in \mathbb{N}$ there are $k$ disjoint $A-B$-paths iff there is no $A-B$ separator of cardinality $\leq k-1$.\\
	\textbf{Definition}\\
	A set of $\{a\}-B$-paths s.t. every two paths only meet in $a$ is called an $a-B$-fan.\\
	\textbf{Corollary}\\
	For every $a \in V(G)$, $B\subseteq V(G)$ there is an $a-B$-fan iff there is no $a-B$-separator $X \subseteq V(G)\setminus \{a\}$ of cardinality $\left|X\right|<k$.\\
	\textbf{Corollary}\\
	For $a,b \in V(G)$ and $ab \notin E(G)$ there are $k$ internally disjoint $a-b$-paths iff there is no $a-b$-separator of cardinality $\left|X\right|<k$.\\
	\textbf{Corollary}\\
	A graph is $k$-connected iff $\forall a,b \in V(G), a\neq b$ there are $k$ internally disjoint paths between them.\\
	\textbf{Definition}\\
	For $F \subseteq E(G)$ we say that $F$ separates $a$ from $b$ if there is no $a-b$-path in $G-F$.\\
	\textbf{Theorem}\\
	For $a,b \in V(G), \; a \neq b$ a graph $G$ has $k$ edge-disjoint $a-b$-paths iff there is no $a-b$-separator $F \subseteq E(G)$ with $\left|F\right|\leq k-1$.\\
	\textbf{Definition}\\
	Let $G$ be a graph. We define the \underline{line graph} $L(G)$ as \begin{itemize}
		\item $V(L(G)) = E(G)$
		\item $E(L(G)) = \{ef: \; e,f \in E(G), \; e \neq f, \; e,f \text{ have an endvertex in common}\}$
	\end{itemize}
	\textbf{Definition}\\
	$H$ is called a \underline{minor} of $G$ if $H$ is obtained from $G$ by performing a sequence of the following operations \begin{itemize}
		\item edge contractions
		\item edge deletions
		\item vertex deletions
	\end{itemize}
	Alternatively: $H$ is a minor of $G$ if $\exists H' \subseteq G$ and $H$ is contained from $H'$ by edge contractions.\\
	Equivalently: $H$ is a minor of $G$ if for every $h \in V(H)$ there is a $V_h \subseteq V(G)$ s.t. \begin{itemize}
		\item all $V_h, V_{h'}$ for $h \neq h'$ are disjoint
		\item $G[V_h]$ is connected
		\item $\forall hh' \in E(H)$ there is an edge betwenn $V_h$ and $V_{h'}$.
	\end{itemize}
	\section{Planarity}
	\textbf{Definition}\\
	Every edge either needs to be a straight line segment or a polygonal arcs which are concatenations of straight line segments. A plane graph is a pair $(V,E)$ s.t. \begin{itemize}
		\item $V$ is finite
		\item $E$ is a finite set of polygonal arcs
		\item no two edges share both endpoints
		\item every endpoint of an edge is a vertex
		\item the interior of any edge contains no vertex and not point of another edge.
	\end{itemize}
	\textbf{Definition} (faces)\\
	Faces are defined via an equivalence relation on $\mathbb{R}^2 \setminus G$ the plane where $G$ has been removed. A face $f$ is defined by $f = \{x,y \in \mathbb{R}^2 \setminus G \text{ s.t. } x\sim y\}$ where $x \sim y$ iff there is a polygonal arc in $\mathbb{R}^2 \setminus G$ that links $x$ to $y$. $F(G)$ is the set of all equivalence classes (or faces) of $G$.\\
	There is exactly one unbounded face, called the outer face.\\
	For a face $f$ we define the face boundary \[G[f] = \{x \in \mathbb{R}^2 \text{ s.t. every neighbourhood of $x$ meets $f$ and } \mathbb{R}^2 \setminus G\}\]
	Notice the following remarks \begin{itemize}
		\item two points $x,y \in f \cup G[f]$ can be linked via a polygonal arc $A \subseteq f \cup \{x,y\}$
		\item a boundary $G[f]$ separates $f$ from the rest 
		\item a plane forest has exactly one face
		\item every cycle has exactly two faces
		\item if there is some cycle then every face boundary contains a cycle
		\item every edge of $G$ lies on the face boundary of at most two faces
		\item every edge on a cycle lies on the face boundary of exactly two faces
		\item if $G$ is 2-connected then every face boundary is a cycle 
		\item $G[f]$ is always a subgraph 
	\end{itemize}
	\subsection{Euler's formula}
	\textbf{Theorem}\\
	Let $G$ be a 2-connected plane graph. Then \[2 = \left|V(G)\right| - \left|E(G)\right| + \left|F(G)\right|\]
	\subsection{Triangulation}
	A plane triangulation is a plane graph such that all face boundaries are triangles. Similarly, a plane quadrangulation is a plane graph such that all faces boundaries are on 4-cycles.\\
	\textbf{Theorem}\\
	Let $G$ be a plane graph with $n$ vertices and $m$ edges. \begin{enumerate}
		\item if $G$ is a plane triangulation, then $m = 3n-6$
		\item if $n \geq 3$, then $m \leq 3n -6$
		\item if $G$ is a plane quadrangulation, then $m = 2n-4$
		\item if $G$ is bipartite, then $m \leq 2n-4$
	\end{enumerate}
	\subsection{$K_5$ and $K_{3,3}$}
	\textbf{Definition}\\
	A graph is \underline{planar} if it is isomorphic to a plane graph.\\
	\textbf{Theorem}\\
	Neither $K_5$ nor $K_{3,3}$ are planar. A graph $G$ is planar iff neither $K_3$ nor $K_{3,3}$ are (topological) minors of $G$. This is equivalent to the case that neither is a subdivision of $G$.\\
	\textbf{Theorem}\\
	Every graph with $\delta(G) \geq 3$ contains a subdivision of $K_4$.
	\section{Colouring}
	\textbf{Definition}\\
	A vertex colouring is defined by an assignment $c: V(G) \to C$ for any set $C$ such that $uv \in E(G) \Rightarrow c(u) \neq c(v)$.\\
	The chromatic number $\chi(G)$ is the smallest number of colours (i.e. $\left|C\right|$) such that $c$ exists.\\
	Laslty, we define the clique number $\omega(G)$ as the size of the largest clique in $G$. Obviously $\chi(G) \geq \omega(G)$.\\
	\textbf{Definition} (Greedy Colouring)\\
	Define $C$ to be a set of the first $\left|C\right|$ integers. Enumerate the vertices of $G$ and define $G_i = G[v_1,...,v_i]$. Colour $v_i$ with the least available integer in $C$.\\
	In the graph $G_i$ we need at most $d_{G_i}(v_i) + 1$ colours. By this remark, it is obvious that this algorithm can be optimized by sorting the $v_i$ descending by their degree.\\
	Thus we can find a new lower bound for the chromatic number: \[\chi(G) \leq max_i\{\delta(G_i)+1\} \leq \max \{\delta(H): H\leq G\}+1\]
	Thus, every graph $G$ contains a sugraph $H$ with $\delta(H) \geq \chi(G)$. We also have the obvious upper bound $\chi(G) \leq \Delta(G)+1$. This inequality is tight for any complete graph and odd cycles.\\
	\textbf{Theorem}\\
	Every planar graph $G$ can be coloured with at most 4 colours.\\
	The proof for this theorem relies heavily on computer work, but there is an easy proof for the 5-colour theorem.\\
	\textbf{Theorem}\\
	A planar graph $G$ that has no triangle subgraph satisfies $\chi(G)\leq 3$.\\
	\textbf{Theorem}\\
	If $G$ is connected and neither complete nor an odd cycle, then $\chi(G) \leq \Delta(G)$.
\end{document}