\documentclass[a4paper, 12pt]{article}

\usepackage{fullpage}
\usepackage[utf8]{inputenc}
\usepackage[english]{babel}
\usepackage{amsmath,amssymb}
\usepackage[explicit]{titlesec}
\usepackage{ulem}
\usepackage[onehalfspacing]{setspace}

\titleformat{\subsection}
{\small}{\thesubsection}{1em}{\uline{#1}}
\begin{document}
	\begin{titlepage} 
		\title{GraTe summary}
		\clearpage\maketitle
		\thispagestyle{empty}
	\end{titlepage}
	\tableofcontents
	\section{Introduction}
	\subsection{Graphs}
	\textbf{Definition} (graph):\\
	$G = (V,E)$, as usual. Other stuff that's well known.\\
	\textbf{Definition} (graph isomorphism problems):\\
	Input: two graphs $G,H$.\\
	Output: is $G \cong H$?\\
	\textbf{Definition} (subgraph):\\
	$H \subseteq G$ is called \underline{induced} if \[\forall uv \in E(G) \text{ with } u,v \in V(H) \Rightarrow uv \in E(H)\]
	\textbf{Conjecture}:\\
	Let $G_1, G_2$ be two finite graphs on at least three vertices. \[\phi: V(G_1) \to V(G_2)\] is a bijection such that \[G_1-v \cong G_2-v \; \forall v\in V(G)\]
	Then $G_1 \cong G_2$.\\
	\textbf{Definition} (unfriendly partition conjecture):\\
	A partition $(A,B)$ of the vertex set $V(G)$ is \underline{unfriendly} if $\forall a \in A:$\[\left|N(a) \cap A \right| \leq \left|N(a) \cap B\right|\] and $\forall b \in B$: \[\left|N(b) \cap B\right| \leq \left|N(b) \cap A\right|\]
	\textbf{Proposition}:\\
	Every finite graph has an unfriendly partition.\\
	\textbf{Definition} (cycle double cover conjecture):\\
	Every graph with only edges that lie in cycles has a cycle double cover, i.e. a family of cycles on the graph s.t. each edge lies on exactly two cycles. Cycles can be repeated.\\
	\textbf{Theorem}:\\
	Every complete graph has a cycle double cover.\\
	\textbf{Definition} (degree of a vertex):\\
	The degree of a node gives the number of edges the node lies on \[d_G(v) = \left|\{uv \in E(G)\}\right|\]
	Minimum degree of a graph $G$ \[\delta(G) = \min_{v \in V(G)} d(v)\]
	The maximum degree $\Delta(G)$ and the average degree $d(G)$ are defined similarly.\\
	The average degree can also e written as \[d(G) = \frac{1}{\left|V(G)\right|} \sum_{v \in V(G)} d_G(v) = \frac{2\left|E(G)\right|}{\left|V(G)\right|}\]
	\textbf{Lemma} (handshake lemma):\\
	The number of vertices of odd degree is always even.\\
	\textbf{Theorem} (Mantel's theorem):\\
	Every graph on $n$ vertices and more than $\frac{n^2}{4}$ edges has a triangle.\\
	\textbf{Definition}:\\
	Mantel's theorem creates graphs without any triangles. For any vertex in such a graph the neighborhood doesn't have any edges. A set $X \subseteq V(G)$ with this property is called an \underline{independent set}. \\
	\section{Paths}
	\textbf{Proposition}:\\
	Let $G$ be a graph. Then $G$ has a path of length $\delta(G)$. If $\delta(G) \geq 2$, then there is a cycle of length $\geq \delta(G)+1$.\\
	
	Now, let's define a relation on the vertices: \[u\sim v :\Leftrightarrow \exists \text{ a $u-v$ path}\]
	\textbf{Lemma}:\\
	$\sim$ is an equivalence relation.
	\begin{itemize}
		\item symmetry: if $u-v$ is a path, $v-u$ is a path as well.
		\item transitivity: if $u-v$ and $v-w$ are paths, $u-w$ is a path.
		\item reflexivity: $u-u$ is a path.
	\end{itemize}
	\textbf{Definition}:\\
	$G$ is connected, if $\forall u,v \in V(G)$ there is a $u-v$ path.\\
	The components of $G$ are induced subgraphs on the equivalence classes of $\sim$.
\end{document}