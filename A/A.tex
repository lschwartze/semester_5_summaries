\documentclass[a4paper, 12pt]{article}

\usepackage{fullpage}
\usepackage[utf8]{inputenc}
\usepackage[ngerman]{babel}
\usepackage{amsmath,amssymb}
\usepackage[explicit]{titlesec}
\usepackage{ulem}
\usepackage[onehalfspacing]{setspace}

\titleformat{\subsection}
{\small}{\thesubsection}{1em}{\uline{#1}}
\begin{document}
	\begin{titlepage} 
		\title{A Zusammenfassung}
		\clearpage\maketitle
		\thispagestyle{empty}
	\end{titlepage}
	\tableofcontents
	\section{Gruppen}
	\subsection{Grundlagen}
	\textbf{Definition} (Gruppe)\\ $(G,\cdot)$ mit \begin{enumerate}
		\item $\cdot$ ist assoziativ; $g_1 \cdot (g_2 \cdot g_3) = (g_1 \cdot g_2) \cdot g_3$
		\item Einselement: $\exists e \in G\; s.t.\; \forall g \in G: e\cdot g = g = g\cdot e$
		\item $\forall g \in G: \; \exists h \in G: \; g\cdot h = e$
	\end{enumerate}
$G$ heißt abelsch falls die Gruppe kommutativ ist.\\
\textbf{Lemma} \begin{enumerate}
	\item Die Identität ist eindeutig
	\item $g_1h = g_2h \Rightarrow g_1 = g_2$
	\item ein inverses ist eindeutig (Kürzungslemma)
\end{enumerate}
\textbf{Definition} (Untergruppe)\\
Für $(G,\cdot)$ eine Gruppe ist mit $H\subset G$ $(H,\cdot)$ eine Untergruppe, wenn \begin{enumerate}
	\item $e \in H$
	\item $h_1, h_2 \in H \; \Rightarrow \; h_1h_2 \in H$
	\item $h \in H \Rightarrow h^{-1} \in H$
\end{enumerate}
\textbf{Definition} (Homomorphismus)\\
$(G,\cdot_G)$ und $(H,\cdot_H)$ Gruppen, dann ist ein Homomorphismus $\varphi: (G,\cdot_G) \to (H,\cdot_H)$ eine Abbildung mit $\forall g_1, g_2 \in G: \; \varphi(g_1 \cdot_G g_2) = \varphi(g_1) \cdot_H \varphi(g_2)$\\
\textbf{Proposition}\\ Sei $\varphi$ ein Gruppenhomomorphismus dann \begin{enumerate}
	\item $\varphi(e_G) = e_H$
	\item $\varphi(g^{-1}) = \varphi(g)^{-1}$
	\item $ker(\varphi) = \{g \in G: \; \varphi(g) = e_H\}$ ist Untergruppe
	\item $\varphi$ ist injektiv $\Leftrightarrow \; ker(\varphi) = \{e_G\}$
\end{enumerate}  
\subsection{$(\mathbb{Z}, +)$}
\textbf{Lemma}\\ $I\subset \mathbb{Z}$ ist Untergruppe, wenn $I = \{0\}$ oder wenn $I = (a) = \{k\cdot a: k \in \mathbb{Z}\}$.\\
\textbf{Definition}\\
Seien $a,b\in\mathbb{Z}$ nicht beide 0. \[ggT(a,b) = \max\{d \in \mathbb{N} \; s.t.\; d|a \text{ und } d|b\}\]
\textbf{Lemma} (Bézout)\\
Es gilt $g = xa+by$ für $x,y \in \mathbb{Z}$.\\
\textbf{Korollar}
\[a|bc \text{ und } ggT(a,b) = 1 \Rightarrow a|c\]
Sei jetzt $n \in \mathbb{N}_{\geq 2}$. Wir definieren eine Äquivalenz $\equiv (mod\; n)$ auf $\mathbb{Z}$ durch \[a\equiv b\; (mod\; n) \Leftrightarrow n|(a-b)\]
\[\mathbb{Z}/n\mathbb{Z} = \{a \;mod\; n: a\in\mathbb{Z}\}\]
Diese Gruppe hat eine additive Gruppenstruktur durch \[\overline{a} + \overline{b} = \overline{a+b}\] und eine multiplikative Struktur durch \[\overline{a}\overline{b} = \overline{ab}\]
\textbf{Theorem}\\
($\mathbb{Z}/n\mathbb{Z}, +, \cdot)$ ist ein Ring und es gibt einen surjektiven Homomorphismus durch $\mathbb{Z} \overset{\pi}{\to} \mathbb{Z}/n\mathbb{Z}$ mit $ker(\pi) = n$.\\
\textbf{Definition}\\
\[\mathbb{Z}/n\mathbb{Z}^* = \{\overline{a} \in \mathbb{Z}/n\mathbb{Z}: \exists b \in \mathbb{Z}/n\mathbb{Z}: \; \overline{a}\overline{b} = 1\} \subset \mathbb{Z}/n\mathbb{Z}\]
\textbf{Lemma}\\
Sei $\overline{a} \in \mathbb{Z}/n\mathbb{Z}$. Dann \[\overline{a} \in (\mathbb{Z}/n\mathbb{Z})^* \Leftrightarrow ggT(a,n) = 1\]
\textbf{Lemma}\\
$\mathbb{Z}/n\mathbb{Z}$ ist Körper $\Leftrightarrow\; n$ ist prim.\\
\textbf{Definition}\\
Für eine Gruppe $G$ und eine Teilmenge $S\subset G$ definiert $\langle S \rangle$ die \underline{kleinste Untergruppe}, die $S$ enthält.\\
\textbf{Proposition}
\[\langle S \rangle = \{s_1^{e_1}, ..., s_N^{e_N}: N \in \mathbb{N}, \; e_i \in \mathbb{Z}, s_i \in S\}\]
\textbf{Definition}\\
Sei $g\in G$. $g$ hat (un)endliche Ordnung, wenn \[\langle g \rangle = \{g^e: e \in \mathbb{Z}\}\]
(un)endlich ist. Es gilt $ord(g) = \left|\langle g \rangle \right|$.\\
Man definiert $\exp_g: \mathbb{Z} \to \langle g \rangle \subseteq G, \; k \mapsto g^k$ ein Homomorphismus.\\
\textbf{Theorem}\\
$\exp_g$ ist injektiv $\Leftrightarrow g$ hat unendliche Ordnung. Wenn die Ordnung $n$ endlich ist, erhalten wir einen Gruppenisomorphismus \[exp_g: \mathbb{Z}/ n\mathbb{Z} \to \langle g\rangle \subseteq G\]\[\overline{a} = a+n\mathbb{Z} \mapsto g^a\]
\textbf{Korollar}
\[ord(g) = \min\{k \in \mathbb{N}: g^k = 1\}\]
\textbf{Korollar}\\
Sei $g \in G$ mit endlicher Ordnung $n$. Dann $ord(g^k) = \frac{n}{ggT(n,k)}$.\\
\textbf{Satz} (Lagrange)\\
Wenn $H\leq G$ und $G$ endlich, dann $\left|H\right| | \left| G \right|$.\\
\textbf{Korollar}\\
Es gilt $ord(g) | \left|G\right|$.
\subsection{zyklische Gruppen}
\textbf{Definition}\\
Eine Gruppe heißt \underline{zyklisch}, wenn sie von nur einem Element generiert wird. Es gibt einen Isomorphismus \[\exp_g: \mathbb{Z} \to G\] wenn $g$ endliche Ordnung hat. $\mathbb{Z}$ hat zwei Generatoren, $\pm1$. Generell hat $G$ zwei Generatoren, $g$ und $g^{-1}$.\\
Wenn $G$ ist zyklisch und der Ordnung $n$, dann \[\exp_g: \mathbb{Z}/n\mathbb{Z} \to G\]
Sei $\overline{a} \in \mathbb{Z}/n\mathbb{Z}$. \begin{enumerate}
	\item $\overline{a}$ generiert $\mathbb{Z}/n\mathbb{Z}$ genau dann, wenn $ord(\overline{a}) = n$.
	\item $\overline{a}$ hat (additive) Ordnung $n$ genau dann, wenn $ggT(a,n)=1$.
	\item $\overline{a}$ hat (additive) Ordnung $n$ $\Leftrightarrow$ $\overline{a}$ hat ein \underline{multiplikatives} Inverses. 
\end{enumerate}
Es folgt $\overline{a}$ generiert $\mathbb{Z}/n\mathbb{Z} \Leftrightarrow \; \overline{a} \in (\mathbb{Z}/n\mathbb{Z})^*$. Eine endliche zyklische Gruppe der Ordnung $n$ besitzt genau $\varphi(n)$ Erzeuger.\\
\textbf{Satz}\\
Sei $G$ zyklisch, dann ist jede Untergruppe von $G$ zyklisch.\\
Wenn $\left|G\right|=n$, und $d|n$, dann existiert genau eine Untergruppe von $G$ der Ordnung $d$.\\
\textbf{Korollar}\\
Sei $G$ zyklisch, von der Ordnung $n$. Dann enthält $G$ für alle $d|n$ genau $\varphi(d)$ Elemente der Ordnung $d$.\\
Es gilt außerdem (für alle Gruppen) \[n = \sum_{d|n} \varphi(d)\]
\textbf{Satz}\\
Sei $G$ endlich. Die folgenden Aussagen sind äquivalent:\begin{enumerate}
	\item $G$ ist zyklisch
	\item $\forall d$ mit $d|n$ gibt es höchstens $d$ Elemente von $G$ deren Ordnung ein Teiler von $d$ ist. 
\end{enumerate}
\textbf{Satz}\\
Sei $K$ ein Körper. Dann ist jede endliche Untergruppe von $K^*$ zyklisch. Wenn $K$ endlich ist, ist also $K^*$ endlich.\\
\textbf{Satz}\\
Seien $G_1, ... G_r$ zyklisch. Dann ist $G_1 \times ... \times G_r$ zyklisch genau dann, wenn die Ordnungen paarweise teilerfremd sind.\\
\underline{Eine Charakterisierung zyklischer Gruppen}\\
Sei $n\geq 2$, $G\cong \mathbb{Z}/n\mathbb{Z}$ eine zyklische Gruppe der Ordnung $n$, dann \begin{itemize}
	\item Für $d|n$ hat $G$ eine eindeutige Untergruppe $H$ der Ordnung $d$. 
	\item $G$ hat $\varphi(n)$ Erzeuger 
\end{itemize}
\textbf{Satz}\\
Ist $G$ eine endliche Gruppe mit \[\forall d|n \text{ gibt es höchstens $d$ Elemente deren Ordnung $d$ teilt}\]
Dann ist $G$ zyklisch\\
\textbf{Korollar}\\
Sei $K$ ein Körper, $G\subset K^*$ endlich. Dann ist $G$ zyklisch. 
\subsection{Permutationsgruppen}
Sei $X$ eine Menge. Man betrachte $Sym(X) = \{f: \; X\to X \text{ bijektiv}\}$. Dann ist $(Sym(X), \circ)$ eine Gruppe. Speziell $S_n = Sym(\{1,...,n\})$.\\
Für Zykelschreibweise siehe EdA.\\
\textbf{Lemma}
\begin{enumerate}
	\item Ist $\sigma$ ein $l$-Zyklus, dann ist $ord(\sigma) = l$
	\item Wenn $\sigma$ Produkt disjunkter Zyklen ist, dann ist $ord(\sigma)$ das kgV der Längen der Zyklen.
\end{enumerate}
Sei $\sigma \in S_n$. Für $f \in \mathbb{Q}[x_1,...,x_n]$ definieren wir \[\sigma(f) = f(x_{\sigma(1)},...,x_{\sigma(n)})\]
Sei \[P = \prod_{1\leq i < j \leq n}(x_i-x_j)\] Dann ist $\sigma(P) = sgn(\sigma)\cdot P = \pm P$.\\
\textbf{Lemma}
\begin{enumerate}
	\item $sgn(\sigma \tau) = sgn(\sigma) sgn(\tau)$. Also $sgn: S_n \to \{\pm1\}$ ist ein Homomorphismus
	\item Ist $\sigma$ eine Transposition $(k\;l)$ mit $k\neq l$, dann $sgn(\sigma) = -1$.
	\item Für einen $l$-Zyklus ist $sgn(\sigma) = (-1)^{l+1}$.
\end{enumerate}
\textbf{Lemma}\\
$A_n = \{\sigma: sgn(\sigma) = 1\}$ ist eine Untergruppe von $S_n$.
Allgemein gilt: \begin{itemize}
	\item $S_n$ wird von seinen Transpositionen erzeugt.
	\item $A_n$ wird von seinen Dreierzyklen erzeugt.
\end{itemize}
\subsection{Gruppenwirkungen}
Sei $G$ eine Gruppe, $X$ eine Menge. Eine \underline{Wirkung} von $G$ auf $X$ ist eine Abbildung \begin{eqnarray*}
	&\rho: G\times X \to X\\
	&(g,x) \mapsto \rho(g,x) = g\cdot x
\end{eqnarray*}
mit \[\begin{cases}
	\forall x \in X: e\cdot x = x\\
	\forall g,h \in G\; x\in X: \; (g\cdot h)\cdot x = g\cdot (h\cdot x)
\end{cases}\]
\textbf{Bemerkung}\\
Gegeben $g \in G$ gilt: \begin{eqnarray*}
	&\sigma_g: X \to X\\
	&x \mapsto g\cdot x
\end{eqnarray*} ist eine Bijektion.
Dann ist die Abbildung \begin{eqnarray*}
	&\sigma: G \to Sym(X)\\
	&g \mapsto \sigma_g
\end{eqnarray*} ein Homomorphismus. Ist $\sigma$ injektiv, so heißt die Wirkung treu (,,faithful'')
\subsection{Bahn-Stabilisator-Satz}
Sei $\rho$ gegeben, dann ist \[x\sim_\rho y \;\Leftrightarrow \; \exists g \in G: y=g\cdot x\] eine Äquivalenzrelation. Eine Äquivalenzklasse ist geschrieben als $G\cdot x = \{g\cdot x: g \in G\}$, genannt \underline{Bahn}. Es gilt daher auch \[X = \bigcup_{i \in I} X_i \;\text{ mit } X_i = G\cdot x_i\] Wenn es nur eine Bahn gibt, heißt die Wirkung \underline{transitiv}.\\
Mit $x \in X$ ist der \underline{Stabilisator} eine Untergruppe $G_x = \{g \in G: g\cdot x = x\}$.\\
\textbf{Satz}\\
Sei $x \in X$. \begin{itemize}
	\item $G/G_x \to G\cdot x$ ist eine Bijektion mit $gG_x \mapsto g\cdot x$.
	\item Wenn $G,X$ endlich sind, dann $\left|G\right| = \left|G_x\right|\cdot \left|G\cdot x\right|$
\end{itemize}
\textbf{Bemerkung}\\
$\left|G_x\right|$ ist nur von der Ordnung von $x$ abhängig. Es gilt \[y \in G_x \Rightarrow \left|G_x\right| = \left|G_y\right|\]
Es gilt sogar \begin{eqnarray*}
	G_x \to G_y\\
	h \mapsto ghg^{-1}
\end{eqnarray*} ist ein Isomorphismus.\\
\textbf{Satz} (Cauchy)\\
Sei $G$ eine endliche Gruppe, $p$ prim, sodass $p | \left|G\right|$. Dann existiert ein $g \in G$ mit $ord(g) = p$.
\subsection{Konjugation}
Es gibt eine andere Wirkung einer Gruppe auf sich selbst. \begin{eqnarray*}
	&G\times G \to G\\
	&(g,h) \mapsto ghg^{-1}
\end{eqnarray*}
Die Bahnen $C(h) = \{ghg^{-1}: g \in G\}$ heißen \underline{Konjugationsklassen} von $G$. Der \underline{Zentralisator} $Z(h) = \{g \in G: ghg^{-1} = h\}$ ist der Stabilisator der Konjugation. Der Bahn-Stabilisator-Satz impliziert eine Bijektion \begin{eqnarray*}
	&G/Z(h) \to C(h)\\
	&gZ(h) \mapsto ghg^{-1}
\end{eqnarray*}
Wenn $G$ endlich ist, dann $G = C\cup C_1 \cup ... \cup C_r$.\\
\textbf{Bemerkung}\\
$D_n$ enthält $2(\lfloor\frac{n}{2}\rfloor +1)$ Konjugationsklassen.\\
\textbf{Satz}\\
Sei $p$ prim und ungerade und $G$ nicht abelsch mit Ordnung $2p$. Dann gilt $G\cong D_{p}$.\\
\textbf{Theorem}\\
Ist $G$ endlich und der Ordnung $p^m$, so gilt $\left|Z(G)\right|>1$.
\subsection{Quotientengruppen}
\textbf{Definition}\\
Sei $H<G$. Wir sagen, $H$ ist \underline{normal}, wenn $gHg^{-1} = H, \; \forall g \in G$. Äquivalent: $gH = Hg$, die Linksnebenklassen entsprechen also für alle $g \in G$ den Rechtsnebenklassen. Außerdem ist $H$ normal genau dann, wenn $ghg^{-1} \in H \; \forall g\in G, h \in H$.\\
\textbf{Bemerkung}\\
Sei $H<G$ gegeben. Die Menge aller Linksnebenklassen sei gegeben durch $G/H = \{gH: g \in G\}$. Analog ist die Menge der Rechtsnebenklassen definiert $H\setminus G$.\\
\textbf{Proposition}\\
Wenn $N<G$ normal ist, dann gibt es eine Gruppenverknüpfung \begin{eqnarray*}
	& \cdot:\; G/N \times G/N \to G/N\\
	& g_1N \cdot g_2N \mapsto g_1g_2N
\end{eqnarray*}
Der kanonische Homomorphismus \begin{eqnarray*}
	& G \overset{\pi}{\to} G/N\\
	& g \mapsto gN
\end{eqnarray*}
ist surjektiv. $G/N$ heißt Quotientengruppe von $G$ unter $N$.\\
\textbf{Korollar}\\
Sei $H<G$. $H$ ist normal in $G$, genau dann, wenn $H$ Kern eines Gruppenhomomorphismus $\varphi: G \to G'$ ist.
\subsection{Isomorphiesatz}
Sei $\varphi: G\to G'$ ein Gruppenhomomorphismus. Sei $N<G$ normal mit $N\subset ker(\varphi)$. Dann $\exists \overline{\varphi}$ wohldefiniert mit \begin{eqnarray*}
	& \overline{\varphi}: G/N \to G\\
	& gN \mapsto \varphi(g)
\end{eqnarray*}
sodass \begin{enumerate}
	\item $im(\overline{\varphi}) = im(\varphi)$
	\item $\overline{\varphi}$ injektiv $\Leftrightarrow \, N = ker(\varphi)$
\end{enumerate}
\textbf{Korollar}\\
Sei $\varphi: G\to G'$ ein Gruppenhomomorphismus. Dann gibt es einen Isomorphismus \[\overline{\varphi}: G/N \to im(\varphi)\] falls $N = ker(\varphi)$\\
\textbf{Korollar}\\
Ein surjektiver Gruppenhomomorphismus $\varphi: G \to G'$ impliziert einen Isomorphismus \[\overline{\varphi}: G/N \to G'\] wenn $N = ker(\varphi)$.\\
\textbf{Theorem}\\
Sei $G$ Gruppe, $N\triangleleft G$. \begin{itemize}
	\item Die Abbildungen \[H<G \to \overline{H} = \underbrace{\{\overline{h}: h \in H\}}_{=\pi(H)} < G/N\] und \[\pi^{-1}(K)<G \leftarrow K<G/N\] liefern gegenseitige Isomorphismen zwischen \[H<G \text{ mit } N<H \leftrightarrow \text{ UG von } G/N\]
	\item $H$ is normal $\Leftrightarrow \pi(H)$ normal für $H<G$ mit $N<H$
	\item ist $H$ normal mit $N<H$, so gilt $G/H \cong (G/N)/(H/N)$
\end{itemize}
\textbf{Proposition}\\
Sei $G$ eine Gruppe und $H<G$ von Index 2, d.h. $\left|G/H\right| = 2$. Dann ist $H$ normal.
\section{Ringe}
\subsection{Grundlagen}
\textbf{Definition}\\
Ein Ring ist eine Menge mit zwei Verknüpfungen $(R,+,\cdot)$. \begin{enumerate}
	\item $(R,+)$ ist eine abelsche Gruppe mit neutralem Element $0\in R$ und Inversem $-a$ für alle $a \in R$.
	\item $\cdot$ ist assoziativ $(a\cdot b) \cdot c = a \cdot (b\cdot c)$ und hat neutrales Element $1 \in R$
	\item $\forall a,b,c \in R$ gilt $a(b+c) = ab + ac$ und $(a+b)c = ac + bc$.
\end{enumerate}
\textbf{Definition}\\
Die Menge von Einheiten $R^*$ von $R$ ist definiert durch \[\{a \in R: \; \exists b \in R: \; a\cdot b = 1 = b\cdot a\}\]
$(R^*, \cdot)$ ist eine Gruppe.\\

Ein Ring heißt \underline{kommutativ}, falls $\forall a,b \in R: \; a\cdot b = b\cdot a$. Zum Beispiel ist $\mathbb{C}$ ein $\mathbb{R}$-Vektorraum der Dimension 2 und ein kommutativer Ring. $\mathbb{C} \times \mathbb{C}$ ist ein $\mathbb{C}$-Vektorraum der Dimension 2 und auch kommutativ.\\
\textbf{Definition}\\
Seien $R$, $S$ Ringe. Ein Ringhomomorphismus $\varphi: R \to S$ ist eine Abbildung von Mengen $R \to S$ mit \begin{enumerate}
	\item $\forall a,b \in R: \; \varphi(a+b) = \varphi(a)+\varphi(b)$
	\item $\forall a,b \in R: \; \varphi(a\cdot b) = \varphi(a) \cdot \varphi(b)$
	\item $\varphi(1_R) = \varphi(1_S)$ 
\end{enumerate}
\textbf{Proposition}\\
Sei $a \in R^*$ und sei $\varphi: R \to S$ ein Ringhomomorphismus. Dann ist $\varphi(a)$ invertierbar und $\varphi(a)^{-1} = \varphi(a^{-1})$.\\
\subsection{Teilbarkeit}
\textbf{Definition}\\
Sei $a \in R$. $a$ heißt Nullteiler, wenn $\exists b \in R\setminus\{0\}$, sodass $ab = 0$. Wenn $0$ der einzige Nullteiler ist, heißt $R$ Integritätsring.\\
\textbf{Bemerkung}\\
Ist $R$ ein Integritätsring und $ab = ac$ mit $a\neq 0$, dann impliziert $a(b-c) = 0$, dass $b=c$.\\
\textbf{Proposition}\\
Sei $R$ ein Integritätsring. Es gibt ein Körper $K$ mit folgenden Eigenschaften: \begin{itemize}
	\item $R$ ist ein Unterring von $K$
	\item $\forall k \in K$ gilt $k = ab^{-1}$ mit $a,b \in R$ und $b\neq 0$
\end{itemize}
So ein Körper heißt \underline{Quotientenkörper} von $R$.
\subsection{Ideale}
Ein Ideal $I$ ist eine ein Unterring $I\subseteq R$ mit folgenden Eigenschaften: \begin{enumerate}
	\item $I$ ist eine Untergruppe von $(R,+)$
	\item $\forall r \in R, \; a \in I$ gilt $ra \in I$
\end{enumerate}
\textbf{Bemerkung}
\begin{itemize}
	\item $\{0\}$ und $R$ sind Ideale
	\item sei $a \in R$. $I = (a) = aR = \{r\cdot a: \; r \in R\}$ ist das Ideal von $a$ generiert.
	\item $I=R$ $\Leftrightarrow$ $a$ ist eine Einheit.
\end{itemize}
\textbf{Definition}\\
Sie $S = \{a_1,...,a_n\} \subset R$ endlich. Dann ist das Ideal $(S) = \{r_1\cdot a_n+...+r_n\cdot a_n\}$ das kleinste Ideal, das $S$ enthält.\\
\textbf{Definition}\\
Ein Hauptidealring ist ein Ring $R$, sodass alle Ideale $I$ der Form $I = (a)$ sind (Hauptideale).
\subsection{Quotientenringe}
\textbf{Definition} (Quotientenringe)\\
Sei $R$ ein Ring, $(a) \subseteq R$ ein Ideal. Wir definieren $R/(a) = \{\overline{r} = r+a = \{r+s : s \in (a)\}\}$. Ist $\varphi: R \to S$ ein Homomorphismus mit $(a) \subseteq ker(\varphi)$, so existiert $\overline{\varphi}: R/(a) \to S$ mit $\overline{r} \mapsto \varphi(r)$.\\

Für einen Körper $K$ ist $K[x]$ ein Ring und für $f \in K[x]$ mit $deg(f) = n$ ist $K[x]/(f) = \{\overline{r}: deg(r) < n\}$. Das folgt aus Division mit Rest und da $g = f\cdot q + r$ mit $deg(r) < deg(f)$ und $r$ ist eindeutig bestimmt.\\

\noindent\textbf{Bemerkung}\\
$a \subseteq R$ ist Ideal $\Leftrightarrow$ $(a)$ ist Kern eines Homomorphismus.\\
\textbf{Proposition}\\
Es gibt für $(a) \subseteq R$ gegenseitig inverse Abbildungen zwischen der Menge der Ideale, die $(a)$ enthalten und der Menge der Ideale von $R/(a)$.\\
\textbf{Definition}\\
$(a) \subseteq R$ heißt \begin{itemize}
	\item \underline{Primideal}, falls $rs \in (a) \Rightarrow r \in (a) \lor s \in (a)$
	\item \underline{maximales Ideal}, falls $(a) \neq R$ und $\not \exists\; I \text{ s.t. } (a) \subsetneq I \subsetneq R$.
\end{itemize} 
\textbf{Satz}\begin{itemize}
	\item $(a)$ prim $\Leftrightarrow$ $R/(a)$ ist Integritätsring
	\item $(a)$ maximal $\Leftrightarrow$ $R/(a)$ ist Körper
\end{itemize}
\textbf{Definition}\\
$a \in R$ heißt \underline{irreduzibel}, falls $a \neq 0, \; a \notin R^*$ und $a = r\cdot s \Rightarrow \; r \in R^* \lor s \in R^*$.\\
$a \in R$ heißt \underline{Primelement}, falls $a \neq 0 \; a \notin R^*$ und $(a)$ ist ein Primideal.\\
\textbf{Satz}\\
Sei $a \in R \setminus \{0\}$ und $R$ ein Hauptidealring. Dann sind äquivalent \begin{enumerate}
	\item $a$ ist irreduzibel
	\item $(a)$ ist prim ($\Leftrightarrow a$ ist prim)
	\item $(a)$ ist maximal
\end{enumerate}
\subsection{Sun Zi}
\textbf{Definition}\\
Seien $I, J$ $R$-Ideale. \begin{itemize}
	\item $I+J = \{r+s, r \in I, s \in J\}$
	\item $I\cdot J = \{\sum_{i=1}^{n} r_is_i\; n \in \mathbb{N}, r \in I, s \in J\}$
\end{itemize}
\textbf{Definition}\\
$I$ und $J$ sind teilerfremd, wenn $I+J = R$.\\
\textbf{Satz}\\
Sei $I+J=R$, dann \begin{enumerate}
	\item $I\cdot J = I\cap J$
	\item Es gilt der Isomorphismus $R/(I\cdot J) \cong R/I \times R/J$
\end{enumerate}
\subsection{Faktorringe}
\textbf{Definition}\\
Seien $a,b \in R$.  \begin{enumerate}
	\item Wir sagen $a \sim b$, wenn $a|b$ und $b|a$ $\Leftrightarrow a = rb$ für $r \in R^*$ $\Leftrightarrow \; (a) = (b)$
	\item Wir sagen $a$ ist prim, wenn $(a)$ ein Primideal ist
\end{enumerate}
\textbf{Satz}\\
Sei $R$ ein Integritätsring, dann ist jedes Primelement von $R$ irreduzibel.\\
\textbf{Proposition}\\
Für $a \in \mathbb{Z}[\sqrt{-5}]$ gilt $a \in R^* \Leftrightarrow \left|\left|a\right|\right| = 1$.\\
\textbf{Definition}\\
Ein Faktorring ist ein Integritätsring mit folgenden Eigenschaften \begin{enumerate}
	\item $a \in R$ mit $a\neq 0$ und $a \notin R^*$, dann $a = \pi_1\cdot ... \cdot \pi_k$ für irreduzible $\pi_i$
	\item Wenn $a = q_1 \cdot ... \cdot q_l$, dann $k = l$ und $\pi_i \sim \alpha q_i$
\end{enumerate}
\textbf{Satz}\\
Sei $R$ ein Integritätsring, sodass (1) gilt. Dann sind äquivalent
\begin{itemize}
	\item $R$ ist ein Faktorring
	\item jedes irreduzible Element von $R$ ist prim.
\end{itemize}
\textbf{Satz}\\
Ein Hauptidealring ist ein Faktorring.\\
\textbf{Korollar}\\
Seien $a,b \in R\setminus\{0\}$ in einem Faktorring. Es gibt einen größten gemeinsamen Teiler $ggT(a,b) = d$ mit $d | a$, $d | b$ und $e | a, \; e | b \Rightarrow e | d$. $d$ ist eindeutig bis auf Assoziiertheit.
\subsection{Gauß' Lemma}
Sei $R$ ein Faktorring und $\mathbb{P}$ eine Menge irreduzibler Elemente in $R$. sodass \begin{itemize}
	\item $\pi, \pi' \in \mathbb{P}$, dann $\pi \sim \pi'$
	\item $q \in R$ irreduzibel, dann $q \sim \pi$ für ein $\pi \in \mathbb{P}$
\end{itemize}
Sei $a = q_1 ... q_s = u \pi_1 ... \pi_s = u\prod_{\pi \in \mathbb{P}} \pi^{e_\pi}$. Dann ist die Faktorisierung eindeutig nach der Wahl von $\mathbb{P}$ und $e_\pi$ ist 0 für eine endliche Anzahl an $\pi$.\\

\noindent\textbf{Bemerkung}\\
Zwei Elemente sind nicht teilerfremd, genau dann, wenn die Exponenten aller Faktoren es erlauben. Mit der Faktorisierung lässt sich das kgV charakterisieren.\\

Seien $a \in R$ und $\pi \in \mathbb{P}$ gegeben. Wir definieren \[v_\pi(a) = \max\{k \in \mathbb{N}: \pi^k | \pi\} \]
\textbf{Lemma}\\
$v_\pi$ kann auf $v_\pi: R \to \mathbb{Z} \cup \infty $ erweitert werden und \begin{enumerate}
	\item $v_\pi(a+b) \geq \min\{v_\pi(a), v_\pi(b)\}$
	\item $v_\pi(a+b) = \min\{v_\pi(a), v_\pi(b)\}$, wenn $v_\pi(a) \neq v_\pi(b)$
	\item $v_\pi(ab) = v_\pi(a) + v_\pi(b)$
\end{enumerate} 
\textbf{Definition}\\
Sei $f \in K[x]$, $K = Q(R)$. Wir definieren $v_\pi(f) = \min_i v_\pi(a_i)$ wobei die $a_i$ die Koeffizienten von $f$ seien. Wir definieren den Inhalt $cont$ von $f$ als \[cont(f) = \prod_{\pi \in \mathbb{P}} \pi^{v_\pi(f)}\]
Ist $cont(f) = 1$, dann ist $f$ primitiv.\\
\textbf{Lemma}\\
Seien $f,g \in K[x]$, $\pi \in \mathbb{P}$. Dann \begin{itemize}
	\item $v_\pi(fg) = v_\pi(f) + v_\pi(g)$
	\item $cont(fg) = cont(f)cont(g)$
	\item sind $f,g$ primitiv, so auch $fg$
\end{itemize}
\textbf{Lemma}\\
Seien $f,g \in R[x]$ und $f$ primitiv. Wenn $f$ $g$ in $K[x]$ teilt, dann auch in $R[x]$.\\
\textbf{Satz}\\
Sei $R$ ein Faktorring und $K$ ein Quotientenkörper von $R$. Dann ist $R[x]$ wieder ein Faktorring. Die Primelemente sind \begin{itemize}
	\item die irreduziblen Elemente von $R$
	\item die primitiven Polynome in $R[x]$, die irreduzibel in $K[x]$ sind.
\end{itemize}
\textbf{Korollar}\\
Für einen Faktorring $R$, Quotientenkörper $K$ und $f \in R[x]$ primitiv, dann $f$ in $R[x]$ irreduzibel genau dann, wenn $f$ in $K[x]$ irreduzibel ist.\\
\textbf{Satz}\\
Sei $R$ ein Faktorring und $f \in R[x]$. Sei $\pi \in R$ prim, sodass der Leitkoeffizient von $f$ nicht durch $\pi$ teilbar ist. Dann ist $f$ irreduzibel in $K[x]$ genau dann, wenn die Reduktion $\overline{f}$ modulo $\pi$ irreduzibel in $K[x]$ ist.\\
\textbf{Satz}\\
Sei $R$ ein Faktorring und $\pi \in R$ prim, sodass \begin{enumerate}
	\item $\pi \not | a_n$
	\item $\pi | a_i$ für $i \in \{0,...,n-1\}$
	\item $\pi^2 \not | a_0$
\end{enumerate}
Dann ist $f$ irreduzibel in $K[x]$.\\
\textbf{Proposition}\\
Sei $f = a_nx^n + ... + a_0 \in R[x]$. Ist $g = x-\alpha$ ein Faktor von $f$ in $K[x]$ mit $\alpha = \frac{r}{s}$ und $r,s$ teilerfremd, dann gilt $s|a_n$ und $r|a_0$.
\section{Körpererweiterungen}
\subsection{Endliche Körpererweiterungen}
Seien $K,L$ Körper und $\varphi: K \to L$ ein Homomorphismus. Dann ist $\varphi$ injektiv.\\
\textbf{Definition}\\
Eine Körpererweiterung $L | K$ ist eine Inklusion von Körpern $K \subset L$.\\
Sind $L_1$, $L_2$ Erweiterungen von einem Körper $K$, so heißt ein Homomorphismus $\varphi: L_1 \to L_2$ ein $k-$Homomorphismus, falls $\varphi |_K = id |_K$.\\
\textbf{Definition}\\
Ist $L|K$ eine Körpererweiterung und ist $S \subseteq L$ so bezeichnen wir mit $K[S]$ den kleinsten Teilring von $L$ der $S$ enthält. und mit $K(S)$ den kleinsten Teilkörper, der $S$ enthält.\\
\textbf{Definition}\\
Man sagt, dass $L|K$ \underline{endlich} ist, wenn $L$ ein endlich dimensionaler Vektorraum über $K$ ist. Der \underline{Grad} $[L:K]$ ist definiert als $\dim_K(L)$. Der Grad ist multiplikativ: \[[M:K] = [M:L][L:K]\]\\
\textbf{Definition}\\
Sei $\alpha \in L$. Wir sagen, dass $\alpha$ \underline{algebraisch} über $K$ ist, falls ein $f \in K[x]\setminus \{0\}$ mit $f(\alpha) = 0$ existiert. $L|K$ heißt algebraisch, falls jedes Element von $L$ algebraisch über $K$ ist.\\
\textbf{Satz}\\
Sei $L|K$ eine Erweiterung und $\alpha \in L|K$ algebraisch. dann gibt es ein eindeutig bestimmtes normiertes, irreduzibles Polynom mit $f(\alpha) = 0$, das Minimalpolynom von $\alpha$. Es gibt außerdem einen Isomorphismus $K(\alpha) \cong K[x]/(f)$. Es gilt $[K(\alpha):K] = \deg(f)$.\\
\textbf{Satz}\\
Folgende Aussagen sind äquivalent: \begin{itemize}
	\item $L|K$ ist endlich
	\item $L|K$ ist endlich erzeugt und algebraisch
\end{itemize}
\textbf{Proposition}\\
Sei $K\subset L \subset M$ mit $L|K$ algebraisch. Ist $\alpha \in M$ algebraisch über $L$, so ist $\alpha$ algebraisch über $K$.\\
\textbf{Satz}\\
Ist $f \in K[x]$ irreduzibel von Grad $n$, so existiert eine einfache Körpererweiterung von Grad $n$, die eine Nullstelle von $f$ enthält und die $L = K(\alpha)$ erfüllt. Zwei solche Erweiterungen sind $K$-isomorph. 
\subsection{Zerfällungskörper}
\textbf{Korollar}\\
Ist $f \in K[x]\setminus\{0\}$, so existiert eine endliche Erweiterung $L$ von $K$ mit folgenden Eigenschaften: \begin{enumerate}
	\item Es gilt $f = c(x-\alpha_1)\dots (x-\alpha_n)$ in $L$
	\item $L = K(\alpha_1,\dots \alpha_n)$
\end{enumerate}
Ein solcher Körper heißt \underline{Zerfällungskörper} von $F$ über $K$. Er ist bis auf Isomorphie eindeutig.\\
\textbf{Lemma}\\
Sei $\sigma: K_1 \to K_2$ ein Körperisomorphismus. \begin{enumerate}
	\item ist $f_1$ in $K_1[x]$ irreduzibel, so auch $f_2$ in $K_2[x]$.
	\item Ist $L_i = K_i[x]/(f_i)$, der zu $fi$ gehörige Körper und sind $\alpha_i$ Nullstellen von $f_i$ in $L_i$ so gibt es einen eindeutigen Isomorphismus $\tau: L_1 \to L_2$.
\end{enumerate} 
\textbf{Lemma}\\
Sei $f \in K[x]$ und seien $L_1, L_2$ Zerfällungskörper von $f$. Seien Zwischenkörper $M_i$ mit K $\subset M_i \subset L_i$ gegeben und sei $\sigma: M_1 \to M_2$ ein $K$-Isomorphismus. Dann existiert ein Isomorphismus $\tau: L_1 \to L_2$ der $\sigma$ fortsetzt.
\subsection{Separierbarkeit}
\textbf{Definition}\\
Sei $K$ ein Körper und $f \in K[x]$ nicht konstant. \begin{itemize}
	\item $f$ heißt \underline{square-free} wenn $\not \exists g \in K[x]$ sodass $g^2 | f$
	\item $f$ heißt \underline{separierbar}, wenn $f$ in jeder Körpererweiterung von $K$ square-free ist. 
\end{itemize}
\textbf{Definition}\\
Sei $K$ ein Körper. Dann ist die Charakteristik $char(K)$ definiert als das kleinste $n$ sodass $\underbrace{1+...+1}_{n} = 0$.\\
\textbf{Lemma}\\
In einem Körper der Charakteristik $p$ gilt \[(a+b)^p = a^p + b^p\]
\textbf{Lemma}\\
Sei $f \in K[x]$, dann ist $f$ separierbar genau dann, wenn \[\gcd(f,f') = 1\]
\textbf{Definition}\\
Sei $L|K$ eine algebraische Erweiterung. \begin{enumerate}
	\item $\alpha \in L$ heißt separierbar über $K$ wenn das Minimalpolynom separierbar über $K$ ist.
	\item Die Erweiterung ist separierbar, wenn jedes Element von $L$ separierbar ist.
\end{enumerate}
\textbf{Theorem}\\
Sei $K$ ein Körper mit Charakteristik 0. Dann gilt \begin{itemize}
	\item Jedes irreduzible Polynom in $K[x]$ ist separierbar.
	\item Jede algebraische Erweiterung ist separierbar.
\end{itemize}
\textbf{Theorem}\\
Sei $L|K$ eine endliche separierbare Erweiterung. Dann existiert ein $\alpha \in L$ sodass $L = K(\alpha)$.\\
\textbf{Theorem}\\
Sei $L|K$ endlich erzeugt. Ist $L|K$ algebraisch, so ist sie endlich.
\subsection{Algebraischer Abschluss}
\textbf{Definition}\\
Ein Körper $K$ heißt algebraisch abgeschlossen, wenn jedes Polynom $f \in K[x]$ eine Nullstelle in $K$ annimmt.\\
\textbf{Lemma}\\
Sei $L|K$ eine Körpererweiterung. Dann gilt \begin{enumerate}
	\item $M = \{a \in L | a\text{ ist algebraisch über } K\}$ ist eine Körpererweiterung von $K$.
	\item Wenn $L$ algebraisch abgeschlossen ist, dann ist $M$ ein algebraischer Abschluss von $K$.
\end{enumerate}
\textbf{Lemma}\\
Sei $I$ ein Ideal eines Ringes $R$. Dann ist $I$ Teilring eines maximalen Ideals.\\
\textbf{Satz}\\
Jeder Körper besitzt einen algebraischen Abschluss.
\subsection{Endlich erzeuge Erweiterungen}
\textbf{Definition}\\
Ein $\alpha \in L$ heißt transzendent über $K$, wenn $\forall f\neq 0 \in K[x]$ gilt $f(\alpha) \neq 0$.\\
\textbf{Definition}\\
Sei $L|K$ gegeben und $S\subset L$.
\begin{itemize}
	\item $S$ ist algebraisch unabhängig über $K$, wenn für eine beliebige Reihenfolge $a_1,...,a_n$ von den Elementen von $S$ gilt, dass der Isomorphismus \begin{align*}
		K[x_1,...,x_n] \to L\\
		f \mapsto f(a_1,...,a_n)
	\end{align*}
	injektiv ist.
	\item Eine Transzendenzbasis von $L|K$ ist eine algebraisch unabhängige Teilmenge $S\subset L$, sodass $L|K(S)$ algebraisch ist.
	\item Eine Körpererweiterung ist rein transzendent, wenn $L=K(S)$ für eine algebraisch unabhängige Menge $S$. 
\end{itemize}
\textbf{Lemma}\\
Sei $L|K$ eine Körpererweiterung und $S \subset $L endlich, sodass $L|K(S)$ algebraisch ist. Dann existiert eine Teilmenge $S'$ von $S$, sodass $S'$ eine Transzendenzbasis von $L$ über $K$ ist.\\

Im Allgemeinen gibt es eine \underline{Transzendenzbasis} $T \subset L$, sodass $K \subset K(T) \subset L$. Hierbei ist $K(T)$ rein transzendent und $L$ algebraisch.\\
\textbf{Bemerkung}\\
Sei als Beispiel $K = \mathbb{R}$ und $L = Q(\mathbb{R}[x,y]/(y^2-(x^3+x+1)))$. Dann sind $\{x\}$ und $\{y\}$ Transzendenzbasen. Es gibt die Abbildung \begin{align*}
	\mathbb{R}[x] & \overset{\phi}{\to} L\\
	x & \mapsto x
\end{align*} 
$\phi$ ist injektiv, dann $y^2-(x^3+x+1)) \cap \mathbb{R}[x] = 0$. D.h. $x$ (und $y$) ist transzendent.\\
Da $L|K$ die transzendenten Elemente $x$, $y$ enthält, ist sie nicht algebraisch. $\{x,y\}$ ist algebraisch abhängig, denn $y^2-(x^3+x+1)= 0$. Daraus folgt, $K\subset K(x)$ (bzw. $K(y)$) sind transzendent aber $K(x) \subset K(x,y)$ sind algebraisch. Die Aussage folgt daraus.\\
\textbf{Lemma}\\
Sei $L|K$ gegeben und $A = \{a_1,...,a_n\}$ eine endliche Teilmenge von $L$, sodass $L|K(A)$ algebraisch ist. Sei $B = \{b_1,...,b_m\}$ eine algebraisch unabhängige Teilmenge von $L$. Dann ist $m \leq n$ und nach Umnummerieren ist \[A' = \{b_1,...,b_m,a_{m+1},...,a_n\}\]
eine Menge, sodass $L|K(A')$ ist algebraisch.\\
\textbf{Satz}\\
Für eine endliche Körpererweiterung gilt \begin{enumerate}
	\item Es gibt eine endliche Transzendenzbasis $S$ von $L|K$
	\item $L|K(S)$ ist endlich
	\item Verschiedene Transzendenzbasen haben die selbe Kardinalität.
\end{enumerate}
\textbf{Definition}\\
Wir definieren den Transzendenzgrad einer Erweiterung als \[tr\deg = \left|S\right|\] für eine Transzendenzbasis $S$.
\section{Endliche Körper}
\textbf{Theorem}\\
Sei eine Primzahlpotenz $p^n$ gegeben. Dann existiert ein endlicher Körper $K$ der Kardinalität $p^n$. Man nehme den Zerfällungskörper von $x^{p^n}-x$ über $F_p$.\\
\textbf{Satz}\\
Seien $K_1$ und $K_2$ zwei Körper mit $\left|K_1\right| = \left|K_2\right| = p^n$. Dann $K_1 \cong K_2$.\\
\textbf{Theorem}\\
Ist $K$ endlich, so ist $K^*$ zyklisch.\\
\textbf{Theorem}\\
Seien $K_1, K_2$ Körper mit $\left|K_i\right| = p^{n_i}$. Dann gilt $K_1 \subset K_2 \Leftrightarrow n_1|n_2$.\\
\section{Moduln}
\subsection{Definitionen}
Sei $R$ ein Ring. Ein $R$-Modul ist eine abelsche Gruppe $(M,+)$ mit skalarer Multiplikation \begin{align*}
	R\times M &\to M\\
	(a,m) &\mapsto a\cdot m
\end{align*}
mit folgenden Eigenschaften: \begin{itemize}
	\item $1\cdot m = m$
	\item $(ab)\cdot m = a(b\cdot m)$
	\item $(a+b) \cdot m = a\cdot m + b \cdot m$
	\item $a(m+m') = a\cdot m + a \cdot m'$ 
\end{itemize}
Ein $R$-Teilmodul ist eine Untergruppe $N$ die abgeschlossen bezüglich skalarer Multiplikation ist, d.h. $r\cdot n \in N$ für alle $n \in N$.\\
Ein Homomorphismus von $R$-Modulen ist ein Gruppenhomomorphismus $\phi$ der $\phi(a\cdot m) = a\cdot \phi(m)$ erfüllt.\\
\textbf{Theorem}\\
Sei $M$ ein $R$-Modul und seien $M_1,...,M_n$ Teilmodule. Dann sind folgende Aussagen äquivalent: \begin{itemize}
	\item Die Abbildung $\phi: M_1 \times ... \times M_n \to M_1+...+M_n$ ist ein Isomorphismus von $R$-Modulen.
	\item Es gilt \[M_i \cap (M_1+...+M_{i-1} + M_{i+1} + ... M_n)\]
	für alle $i \in [n]$.
	\item Jedes $x \in M$ hat eine eindeutige Darstellung $m = x_1+...+x_n$ für $x_i \in M_i$.
\end{itemize} 
\textbf{Definition}\\
Sei $M$ ein $R$-Modul. \begin{itemize}
	\item $M$ heißt noethersch, falls jedes $R$-Untermodul von $M$ endlich erzeugt ist über $R$.
	\item $R$ heißt noethersch falls jedes Ideal in $R$ endlich erzeugt ist.
\end{itemize}
\textbf{Satz}\\
Ist $R$ ein noetherscher Ring, dann ist $R[x]$ noethersch.\\
\textbf{Satz}\\
Ist $R$ noethersch, so gilt \begin{enumerate}
	\item jedes endlich erzeugte $R$-Modul ist noethersch
	\item jeder Ring der Form $S = R[x_1,...,x_n]/R$ ist noethersch.
\end{enumerate}
\textbf{Definition}\\
Sei $R$ ein Ring, $M$ ein $M$-Modul, $S = (m_i)_{i\in I}$. Dann heißt $S$ linear unabhängig falls \[\sum_{i \in I}a_i m_i = 0 \Rightarrow a_i = 0\] wenn $a_i \in R$ null für fast alle $i \in I$ ist. Eine Basis von $M$ ist ein linear unabhängiges Erzeugersystem.\\
\textbf{Lemma}\\
\begin{enumerate}
	\item Sei $A \in M_{n,n}(R)$. Dann ist $A \in Gl_n(R) \Leftrightarrow \det(A) \in R^*$
	\item Seien $A \in M_{m,n}(R), \; B \in M_{n,m}(R)$, sodass $AB = I_m$. Dann ist $n \geq m$ und sogar $AB=BA=I_m$ falls $n=m$.
\end{enumerate}
\textbf{Definition}\\
Ein Modul heißt frei, falls $M \cong \oplus_{i \in I} R$. Dies ist genau dann der Fall, wenn $M$ eine Basis hat.\\
\textbf{Satz}\\
Ist $M$ frei und endlich erzeugt, so hat $M$ eine endliche Basis $(m_1,...,m_n)$, deren Kardinalität eindeutig bestimmt ist.
\subsection{Torsion}
\textbf{Definition}\\
Das Torsion-Untermodul ist definiert als \[M_{tor} = \{m \in M: \; \exists a \in R\setminus \{0\}: a\cdot m = 0\}\]
\textbf{Satz}\\
Sei $A \in M_{n,n}(R)$, dann ist $c_A(t) = \det(A-tI_n) \in R[t]$. Es gilt $C_A(A) = 0$.\\
\textbf{Satz}\\
Sei $M$ ein endlich erzeugtes freies Modul über $R$ und sei $N \subseteq M$ ein freies $R$-Modul. Dann ist $N$ endlich generiert und $rk(N)\leq rk(M)$.\\
\textbf{Bemerkung}\\
Ein nicht freies Untermodul eines Moduls ist nicht zwingend endlich erzeugt.
\subsection{Geometrische Anwendungen}
\textbf{Definition}\\
Sei $B$ ein topologischer Raum. Ein Vektor Bündel von rang $n$ über $B$ ist eine stetige Abbildung \[\pi: E \to B\] wobei $E$ ein topologischer Raum ist. $\pi$ erfüllt folgende Eigenschaften: \begin{itemize}
	\item $\forall x \in B$ hat $E_x = \pi^{-1}(x)$ die Struktur eines $n$ dimensionalen Vektorraums
	\item $\forall x \in B$ existiert eine Umgebung $U$ und ein Homeomorphismus $\phi: B \times \mathbb{R}^n \to \pi^{-1}(U)$ sodass $\phi \circ \pi$ die kanonische Projektion ist und $\phi$ induziert einen Homeomorphismus von Vektorräumen.
\end{itemize}
\textbf{Lemma}\\
Sei $\pi: E \to B$ ein Vektor Bündel und sei $R = \mathcal{C}(B,\mathbb{R})$. Dann ist $\pi$ trivial genau dann, wenn $M_\pi$ ist ein freies $R$-Modul. 
\subsection{Moduln über Hauptidealringe}
\textbf{Definition}\\
Sei $R$ ein Hauptidealring und $M$ ein $R$-Modul. Wir definieren \[ann(M) = \{r \in R: \forall m \in M: r\cdot m = 0\}\]
Diese Menge ist ein Ideal.
\textbf{Satz}\\
Sei $R$ ein Hauptidealring und $M$ ein freies $R$-Modul von Rang $r$. Ist $N\subset M$ ein Untermodul, dann ist $N$ frei von Rang höchstens $r$.\\
\textbf{Lemma}\\
Ist $R$ ein Hauptidealring und $M$ ein endlich generiertes $R$-Modul mit $M_{tor} = 0$, dann ist $M$ frei.\\
\textbf{Theorem}\\
Sei $R$ ein Hauptidealring und $M$ ein endlich erzeugtes $R$-Modul. \begin{itemize}
	\item $M/M_{tor}$ ist frei
	\item Es gibt ein $F \subset M$ frei mit endlichem Rang, sodass $M = F\oplus M_{tor}$
\end{itemize}
\end{document}