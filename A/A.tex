\documentclass[a4paper, 12pt]{article}

\usepackage{fullpage}
\usepackage[utf8]{inputenc}
\usepackage[ngerman]{babel}
\usepackage{amsmath,amssymb}
\usepackage[explicit]{titlesec}
\usepackage{ulem}
\usepackage[onehalfspacing]{setspace}

\titleformat{\subsection}
{\small}{\thesubsection}{1em}{\uline{#1}}
\begin{document}
	\begin{titlepage} 
		\title{A Zusammenfassung}
		\clearpage\maketitle
		\thispagestyle{empty}
	\end{titlepage}
	\tableofcontents
	\section{EdA Zeugs}
	\subsection{Gruppen}
	\textbf{Definition} (Gruppe):\\ $(G,\cdot)$ mit \begin{enumerate}
		\item $\cdot$ ist assoziativ; $g_1 \cdot (g_2 \cdot g_3) = (g_1 \cdot g_2) \cdot g_3$
		\item Einselement: $\exists e \in G\; s.t.\; \forall g \in G: e\cdot g = g = g\cdot e$
		\item $\forall g \in G: \; \exists h \in G: \; g\cdot h = e$
	\end{enumerate}
$G$ heißt abelsch falls die Gruppe kommutativ ist.
\textbf{Lemma} \begin{enumerate}
	\item Die Identität ist eindeutig
	\item $g_1h = g_2h \Rightarrow g_1 = g_2$
	\item ein inverses ist eindeutig (Kürzungslemma)
\end{enumerate}
\textbf{Definition} (Untergruppe):\\
Für $(G,\cdot)$ eine Gruppe ist mit $H\subset G$ $(H,\cdot)$ eine Untergruppe, wenn \begin{enumerate}
	\item $e \in H$
	\item $h_1, h_2 \in H \; \Rightarrow \; h_1h_2 \in H$
	\item $h \in H \Rightarrow h^{-1} \in H$
\end{enumerate}
\textbf{Definition} (Homomorphismus):\\
$(G,\cdot_G)$ und $(H,\cdot_H)$ Gruppen, dann ist ein Homomorphismus $\phi: (G,\cdot_G) \to (H,\cdot_H)$ eine Abbildung mit $\forall g_1, g_2 \in G: \; \phi(g_1 \cdot_G g_2) = \phi(g_1) \cdot_H \phi(g_2)$\\
\textbf{Proposition} Sei $\phi$ ein Gruppenhomomorphismus dann \begin{enumerate}
	\item $\phi(e_G) = e_H$
	\item $\phi(g^{-1}) = \phi(g)^{-1}$
	\item $ker(\phi) = \{g \in G: \; \phi(g) = e_H\}$ ist Untergruppe
	\item $\phi$ ist injektiv $\Leftrightarrow \; ker(\phi) = \{e_G\}$
\end{enumerate}  
\subsection{$(\mathbb{Z}, +)$}
\textbf{Lemma}:\\ $I\subset \mathbb{Z}$ ist Untergruppe, wenn $I = \{0\}$ oder wenn $I = (a) = \{k\cdot a: k \in \mathbb{Z}\}$.
\textbf{Definition}\\
Seien $a,b\in\mathbb{Z}$ nicht beide 0. \[ggT(a,b) = \max\{d \in \mathbb{N} \; s.t.\; d|a \text{ und } d|b\}\]
\textbf{Lemma} (Bézout):\\
Es gilt $g = xa+by$ für $x,y \in \mathbb{Z}$.\\
\textbf{Korollar}:
\[a|bc \text{ und } ggT(a,b) = 1 \Rightarrow a|c\]
Sei jetzt $n \in \mathbb{N}_{\geq 2}$. Wir definieren eine Äquivalenz $\equiv (mod\; n)$ auf $\mathbb{Z}$ durch \[a\equiv b\; (mod\; n) \Leftrightarrow n|(a-b)\]
\[\mathbb{Z}/n\mathbb{Z} = \{a \;mod\; n: a\in\mathbb{Z}\}\]
Diese Gruppe hat eine additive Gruppenstruktur durch \[\overline{a} + \overline{b} = \overline{a+b}\] und eine multiplikative Struktur durch \[\overline{a}\overline{b} = \overline{ab}\]
\textbf{Theorem}:\\
($\mathbb{Z}/n\mathbb{Z}, +, \cdot)$ ist ein Ring und es gibt einen surjektiven Homomorphismus durch $\mathbb{Z} \overset{\pi}{\to} \mathbb{Z}/n\mathbb{Z}$ mit $ker(\pi) = n$.\\
\textbf{Definition}:\\
\[\mathbb{Z}/n\mathbb{Z}^* = \{\overline{a} \in \mathbb{Z}/n\mathbb{Z}: \exists b \in \mathbb{Z}/n\mathbb{Z}: \; \overline{a}\overline{b} = 1\} \subset \mathbb{Z}/n\mathbb{Z}\]
\textbf{Lemma}:\\
Sei $\overline{a} \in \mathbb{Z}/n\mathbb{Z}$. Dann \[\overline{a} \in (\mathbb{Z}/n\mathbb{Z})^* \Leftrightarrow ggT(a,n) = 1\]
\textbf{Lemma}:\\
$\mathbb{Z}/n\mathbb{Z}$ ist Körper $\Leftrightarrow\; n$ ist prim.\\
\textbf{Definition}:\\
Für eine Gruppe $G$ und eine Teilmenge $S\subset G$ definiert $\langle S \rangle$ die \underline{kleinste Untergruppe}, die $S$ enthält.\\
\textbf{Proposition}:
\[\langle S \rangle = \{s_1^{e_1}, ..., s_N^{e_N}: N \in \mathbb{N}, \; e_i \in \mathbb{Z}, s_i \in S\}\]
\textbf{Definition}:\\
Sei $g\in G$. $g$ hat (un)endliche Ordnung, wenn \[\langle g \rangle = \{g^e: e \in \mathbb{Z}\}\]
(un)endlich ist. Es gilt $ord(g) = \left|\langle g \rangle \right|$.\\
Man definiert $\exp_g: \mathbb{Z} \to \langle g \rangle \subseteq G, \; k \mapsto g^k$ ein Homomorphismus.\\
\textbf{Theorem}:\\
$\exp_g$ ist injektiv $\Leftrightarrow g$ hat unendliche Ordnung. Wenn die Ordnung $n$ endlich ist, erhalten wir einen Gruppenisomorphismus \[exp_g: \mathbb{Z}/ n\mathbb{Z} \to \langle g\rangle \subseteq G\]\[\overline{a} = a+n\mathbb{Z} \mapsto g^a\]
\textbf{Korollar}:
\[ord(g) = \min\{k \in \mathbb{N}: g^k = 1\}\]
\textbf{Korollar}:\\
Sei $g \in G$ mit endlicher Ordnung $n$. Dann $ord(g^k) = \frac{n}{ggT(n,k)}$.\\
\textbf{Satz} (Lagrange):\\
Wenn $H\leq G$ und $G$ endlich, dann $\left|H\right| | \left| G \right|$.\\
\textbf{Korollar}:\\
Es gilt $ord(g) | \left|G\right|$.
\subsection{zyklische Gruppen}
\textbf{Definition}:\\
Eine Gruppe heißt \underline{zyklisch}, wenn sie von nur einem Element generiert wird. Es gibt einen Isomorphismus \[\exp_g: \mathbb{Z} \to G\] wenn $g$ endliche Ordnung hat. $\mathbb{Z}$ hat zwei Generatoren, $\pm1$. Generell hat $G$ zwei Generatoren, $g$ und $g^{-1}$.\\
Wenn $G$ ist zyklisch und der Ordnung $n$, dann \[\exp_g: \mathbb{Z}/n\mathbb{Z} \to G\]
Sei $\overline{a} \in \mathbb{Z}/n\mathbb{Z}$. \begin{enumerate}
	\item $\overline{a}$ generiert $\mathbb{Z}/n\mathbb{Z}$ genau dann, wenn $ord(\overline{a}) = n$.
	\item $\overline{a}$ hat (additive) Ordnung $n$ genau dann, wenn $ggT(a,n)=1$.
	\item $\overline{a}$ hat (additive) Ordnung $n$ $\Leftrightarrow$ $\overline{a}$ hat ein \underline{multiplikatives} Inverses. 
\end{enumerate}
Es folgt $\overline{a}$ generiert $\mathbb{Z}/n\mathbb{Z} \Leftrightarrow \; \overline{a} \in (\mathbb{Z}/n\mathbb{Z})^*$. Eine endliche Gruppe zyklische Gruppe der Ordnung $n$ besitzt genau $\phi(n)$ Erzeuger.\\
\textbf{Satz}:\\
Sei $G$ zyklisch, dann ist jede Untergruppe von $G$ zyklisch.\\
Wenn $\left|G\right|=n$, und $d|n$, dann existiert genau eine Untergruppe von $G$ der Ordnung $d$.\\
\textbf{Korollar}:\\
Sei $G$ zyklisch, von der Ordnung $n$. Dann enthält $G$ für alle $d|n$ genau $\phi(d)$ Elemente der Ordnung $d$.\\
Es gilt außerdem (für alle Gruppen) \[n = \sum_{d|n} \phi(d)\]
\textbf{Satz}:\\
Sei $G$ endlich. Die folgenden Aussagen sind äquivalent:\begin{enumerate}
	\item $G$ ist zyklisch
	\item $\forall d$ mit $d|n$ gibt es höchstens $d$ Elemente von $G$ deren Ordnung ein Teiler von $d$ ist. 
\end{enumerate}
\textbf{Satz}:\\
Sei $K$ ein Körper. Dann ist jede endliche Untergruppe von $K^*$ zyklisch. Wenn $K$ endlich ist, ist also $K^*$ endlich.\\
\textbf{Satz}:\\
Seien $G_1, ... G_r$ zyklisch. Dann ist $G_1 \times ... \times G_r$ zyklisch genau dann, wenn die Ordnungen paarweise teilerfremd sind.\\
\underline{Eine Charakterisierung zyklischer Gruppen}\\
Sei $n\geq 2$, $G\cong \mathbb{Z}/n\mathbb{Z}$ eine zyklische Gruppe der Ordnung $n$, dann \begin{itemize}
	\item Für $d|n$ hat $G$ eine eindeutige Untergruppe $H$ der Ordnung $d$. 
	\item $G$ hat $\phi(n)$ Erzeuger 
\end{itemize}
\textbf{Satz}
Ist $G$ eine endliche Gruppe mit \[\forall d|n \text{ gibt es höchstens $d$ Elemente deren Ordnung $d$ teilt}\]
Dann ist $G$ zyklisch\\
\textbf{Korollar}:\\
Sei $K$ ein Körper, $G\subset K^*$ endlich. Dann ist $G$ zyklisch. 
\subsection{Permutationsgruppen}
Sei $X$ eine Menge. Man betrachte $Sym(X) = \{f: \; X\to X \text{ bijektiv}\}$. Dann ist $(Sym(X), \circ)$ eine Gruppe. Speziell $S_n = Sym(\{1,...,n\})$.\\
Für Zykelschreibweise siehe EdA.\\
\textbf{Lemma}:
\begin{enumerate}
	\item Ist $\sigma$ ein $l$-Zyklus, dann ist $ord(\sigma) = l$
	\item Wenn $\sigma$ Produkt disjunkter Zyklen ist, dann ist $ord(\sigma)$ das kgV der Längen der Zyklen.
\end{enumerate}
Sei $\sigma \in S_n$. Für $f \in \mathbb{Q}[x_1,...,x_n]$ definieren wir \[\sigma(f) = f(x_{\sigma(1)},...,x_{\sigma(n)})\]
Sei \[P = \prod_{1\leq i < j \leq n}(x_i-x_j)\] Dann ist $\sigma(P) = sgn(\sigma)\cdot P = \pm P$.\\
\textbf{Lemma}:
\begin{enumerate}
	\item $sgn(\sigma \tau) = sgn(\sigma) sgn(\tau)$. Also $sgn: S_n \to \{\pm1\}$ ist ein Homomorphismus
	\item Ist $\sigma$ eine Transposition $(k\;l)$ mit $k\neq l$, dann $sgn(\sigma) = -1$.
	\item Für einen $l$-Zyklus ist $sgn(\sigma) = (-1)^{l+1}$.
\end{enumerate}
\textbf{Lemma}:\\
$A_n = \{\sigma: sgn(\sigma) = 1\}$ ist eine Untergruppe von $S_n$.
Allgemein gilt: \begin{itemize}
	\item $S_n$ wird von seinen Transpositionen erzeugt.
	\item $A_n$ wird von seinen Dreierzyklen erzeugt.
\end{itemize}
\subsection{Gruppenwirkungen}
Sei $G$ eine Grupp4, $X$ eine Menge. Eine \underline{Wirkung} von $G$ auf $X$ ist eine Abbildung \begin{eqnarray*}
	&\rho: G\times X \to X\\
	&(g,x) \mapsto \rho(g,x) = g\cdot x
\end{eqnarray*}
mit \[\begin{cases}
	\forall x \in X: e\cdot x = x\\
	\forall g,h \in G\; x\in X: \; (g\cdot h)\cdot x = g\cdot (h\cdot x)
\end{cases}\]
\textbf{Bemerkung}:\\
Gegeben $g \in G$ gilt: \begin{eqnarray*}
	&\sigma_g: X \to X\\
	&x \mapsto g\cdot x
\end{eqnarray*} ist eine Bijektion.
Dann ist die Abbildung \begin{eqnarray*}
	&\sigma: G \to Sym(X)\\
	&g \mapsto \sigma_g
\end{eqnarray*} ein Homomorphismus. Ist $\sigma$ injektiv, so heißt die Wirkung treu (,,faithful'')
\subsection{Bahn-Stabilisator-Satz}
Sei $\rho$ gegeben, dann ist \[x\sim_\rho y \;\Leftrightarrow \; \exists g \in G: y=g\cdot x\] eine Äquivalenzrelation. Eine Äquivalenzklasse ist geschrieben als $G\cdot x = \{g\cdot x: g \in G\}$, genannt \underline{Bahn}. Es gilt daher auch \[X = \bigcup_{i \in I} X_i \;\text{ mit } X_i = G\cdot x_i\] Wenn es nur eine Bahn gibt, heißt die Wirkung \underline{transitiv}.\\
Mit $x \in X$ ist der \underline{Stabilisator} eine Untergruppe $G_x = \{g \in G: g\cdot x = x\}$.\\
\textbf{Satz}:\\
Sei $x \in X$. \begin{itemize}
	\item $G/G_x \to G\cdot x$ ist eine Bijektion mit $gG_x \mapsto g\cdot x$.
	\item Wenn $G,X$ endlich sind, dann $\left|G\right| = \left|G_x\right|\cdot \left|G\cdot x\right|$
\end{itemize}
\textbf{Bemerkung}:\\
$\left|G_x\right|$ ist nur von der Ordnung von $x$ abhängig. Es gilt \[y \in G_x \Rightarrow \left|G_x\right| = \left|G_y\right|\]
Es gilt sogar \begin{eqnarray*}
	G_x \to G_y\\
	h \mapsto ghg^{-1}
\end{eqnarray*} ist ein Isomorphismus.\\
\textbf{Satz} (Cauchy):\\
Sei $G$ eine endliche Gruppe, $p$ prim, sodass $p | \left|G\right|$. Dann existiert ein $g \in G$ mit $ord(g) = p$.
\subsection{Konjugation}
Es gibt eine andere Wirkung einer Gruppe auf sich selbst. \begin{eqnarray*}
	&G\times G \to G\\
	&(g,h) \mapsto ghg^{-1}
\end{eqnarray*}
Die Bahnen $C(h) = \{ghg^{-1}: g \in G\}$ heißen \underline{Konjugationsklassen} von $G$. Das \underline{Zentralisator} $Z(h) = \{g \in G: ghg^{-1} = h\}$ ist der Stabilisator der Konjugation. Der Bahn-Stabilisator-Satz impliziert eine Bijektion \begin{eqnarray*}
	&G/Z(h) \to C(h)\\
	&gZ(h) \mapsto ghg^{-1}
\end{eqnarray*}
Wenn $G$ endlich ist, dann $G = C\cup C_1 \cup ... \cup C_r$.\\
\textbf{Satz}:\\
Sei $p$ prim und ungerade und $G$ nicht abelsch mit Ordnung $2p$. Dann gilt $G\cong D_{2p}$.
\end{document}