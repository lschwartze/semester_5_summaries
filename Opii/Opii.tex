\documentclass[a4paper, 12pt]{article}

\usepackage{fullpage}
\usepackage[utf8]{inputenc}
\usepackage[english]{babel}
\usepackage{amsmath,amssymb}
\usepackage[explicit]{titlesec}
\usepackage{ulem}
\usepackage[onehalfspacing]{setspace}
\usepackage{algpseudocode}

\titleformat{\subsection}
{\small}{\thesubsection}{1em}{\uline{#1}}
\begin{document}
	\begin{titlepage} 
		\title{Opii summary}
		\clearpage\maketitle
		\thispagestyle{empty}
	\end{titlepage}
	\tableofcontents
	\section{Introduction}
	\subsection{Turing Machine}
	\textbf{Definition} (alphabet)\\
	An alphabet $A$ is a set containing at least two symbols excluding $\sqcup$. $A^n$ is the set of all words with length $n$ over $A$. By convention $A^0 = \{\sqcup\}$. Additionally \[A^* = \bigcup_{n \in \mathbb{N}_0} A^n\] A language $L$ is a subset of $A^*$. A word is an element of $L$.\\
	\textbf{Definition} (Turing machine)\\
	Let $\overline{A} = A \cup \{\sqcup\}$. A Turing machine is a function \[\phi : \underbrace{\{0,...,N\}}_{\text{state}} \times \underbrace{\overline{A}}_{\text{symbol}} \to \underbrace{\{-1, 0, ..., N\}}_{\text{new state}} \times \underbrace{\overline{A}}_{\text{new symbol}} \times \underbrace{\{-1, 0, 1\}}_{\text{left, stay, right}}\]
	for any $N \geq 0$.\\
	The computation of $\phi$ for input $x \in A^*$ is a sequence of triplets \[(n^{(i)}, s^{(i)}, \pi^{(i)})\]
	for some $i \in \mathbb{N}_0$ with \begin{itemize}
		\item $n^{(i)} \in \{-1, 0, ..., N\}$
		\item $s^{(i)} \in \overline{A}^\mathbb{Z}$
		\item $\pi^{(i)} \in \mathbb{Z}$ 
	\end{itemize}
	For $i = 0$ we have:\begin{itemize}
		\item $n^0 = 0$ (initial state)
		\item $s_j^0 = \begin{cases}
			x_j, & 1\leq j \leq length(x)\\
			\sqcup, & j \leq 0, \text{ or } j\geq length(x)+1
		\end{cases}$
		\item $\pi^0 = 1$
	\end{itemize} 
	if $n^{(i)} = -1$, the computation ends. We set $time(\phi, x) := i$ and $output(\phi, x) := (s_1^{(i)}, ..., s_k^{(i)})$ where $k = \min\{j \in \mathbb{N}_0 : s_j^{(i)} = \sqcup\} -1$.\\
	\textbf{Definition} (computational problem)\\
	A pair $(X, R)$ where $X\subset A^*$ is a language and $R\subset X \times A^*$ is a relation such that $\forall x \in X: \; \exists y \in A^*$ with $(x,y) \in R$ is called a computational problem. $\phi$ solves a problem if $time(\phi, x) < \infty$ and $(x, output(\phi, x)) \in R$. If $time(\phi, x) < p(length(x))$ for a $p \in \mathbb{P}_n$ then $\phi$ is a polynomial turing machine.
	
	\section{Complexity and NP-completeness}
	\textbf{Definition} (Decision problem, class P)\\
	A decision problem $P$ is a pair $(X,Y)$ where $X$ is a language that is decidable in polynomial time and $Y\subseteq X$. The elements of $Y$ are \underline{yes-instances} of $P$ and the elements of $X\setminus Y$ are \underline{no-instances} of $P$. An algorithm for $P$ (a Turing machine) computes the function $f: X\to \{0,1\}$ where $f(x) = 1$ iff $x \in Y$. If there is an algorithm for $P$ that works in polynomial time, $P$ belongs to the class \textbf{P}.\\
	\textbf{Definition} (NP and CO-NP classes)\\
	A problem $P = (X,Y)$ belongs to \textbf{NP} if there is a polynomial $p$ and a decision problem $P' = (X',Y')$ in \textbf{P} where \[X' = \{x\# c \;|\; x \in X \text{ and } c \in \{0,1\}^{\lfloor p(size(x))\rfloor}\}\]
	such that \[Y = \{x \in X \; |\; \exists c \in \{0,1\}^{\lfloor p(size(x))\rfloor} \text{ s.t. } x\#c \in Y'\}\]
	The string $c$ is a certificate for $x$. An algorithm for $P'$ checks the certificate. A problem belongs to \textbf{co-NP} iff $(X, X\setminus Y)$ is in \textbf{P}.\\
	\textbf{Theorem}
	\begin{itemize}
		\item $P \subseteq NP \;\cap\; co-NP$
		\item \texttt{Hamiltonian cycle} is in $NP$.
		\item \texttt{Integer linear inequalities} is in $NP$.
	\end{itemize}
	\textbf{Definition} (NP-completeness)\\
	A decision problem $P_1 = (X_1, Y_1)$ can be polynomially transformed to a decision problem $P_2 = (X_2,Y_2)$ if $\exists f: X_1 \to X_2$ computable in polynomial time s.t. \[f(x) \in Y_2 \Leftrightarrow x \in Y_1 \; \forall x \in X_1\]
	A problem $P$ is \textbf{NP-complete} if all problems in \textbf{NP} can be transformed to $P$.
	\subsection{Boolean Formulas}
	We focus on formulas in conjunctive form (KNF). $n$-SAT is defined as the problem to find values for each literal s.t. the formula evaluates to true. Each clause has $n$ literals.\\
	Let $x_1,...,x_n$ be variables, $x_i \in \{0,1\}$. We write $\overline{x_i}$ for a negative literal. For a formula in disjunctive form (DNF) an assignment that evaluates to true can always be found in polynomial time regardless of $n$.\\
	\textbf{Theorem}\\
	\texttt{2-SAT} $\in$ \textbf{P}.\\
	\underline{Proof}\\
	Let $f$ be given as CNF. First, we can remove all clauses of length one by assigning the variable true or false so that this clause evaluates to true. Additionally, no subformula of the form $x_i \land \overline{x_i}$ can exist, because these set $f$ to false immediately. We construct a digraph $D$ with vertices $V(D) = \{x_1,...,x_n\} \cup \{\overline{x_1},...,\overline{x_n}\}$. The edges $(\overline{x}, y)$ and $(\overline{y}, x)$ correspond to $\overline{x} \implies y$ and $\overline{y} \implies x$ for a clause $(x\lor y)$. Now, we identify the strong components of $D$, $D_1,...,D_c$ which are the maximal components such that all pairs $(u,v)$ in $D_i$ are connected by a $u-v$-path and a $v-u$-path in $D_i$. Finding $D_1,...,D_c$ can be done in polynomial time. By contracting each $D_i$ to a node $u_i$ we get a new acyclic graph $\overset{\sim}{D}$ with a topological order: If $u_1,...,u_k$ is a topological order, then there is no arc $(u_j,u_i)$ with $j>i$.\\
	We claim the following: $f$ is satisfiable iff no strong component has a variable $x_i$ and its negation $\overline{x_i}$.\\
	$\Rightarrow$: if both belong to a strong component, $x$ can't be set to true nor false, the statement follows.\\
	$\Leftarrow$: Suppose $x_i \in D_r$ and $\overline{x_i} \in D_s$. If $r<s$ in the topological set $x_i = 0$ and vice versa.\\
	\textbf{Theorem}\\
	\texttt{3-SAT} $\in$ \textbf{NPC}\\
	\underline{Proof}\\
	Since we can check any assignement of truth values in polynomial time, it is obvious that \texttt{3-SAT} is in \textbf{NP}. Since we already know that \texttt{Satisfiability} is in \textbf{NPC} we construct a transformation from a formula $f_{SAT}$ to $f_{3SAT}$.\\
	Let $f_{SAT} = C_1 \land ... \land C_m$. \begin{itemize}
		\item if $\left|C_i\right| = 3$ we can leave this clause in $f_{3SAT}$.
		\item if $\left|C_i\right| \geq 4$ we introduce $k-3$ variables $y_1,...,y_{k-3}$ and $k-2$ clauses \[(x_1 \lor x_2 \lor y_1) \land (\overline{y_1} \lor x_3 \lor y_2) \land ... \land (\overline{y_{k-4}} \lor x_{k-2} \lor y_{k-3}) \land (\overline{y_{k-3}} \lor x_{k-1} \lor x_k)\]
		Note that: \begin{itemize}
			\item{} If the assignment for $x_i$ makes $(x_1 \lor ... \lor x_n)$ true, then each if the $k-2$ clauses will be true.
			\item If all $k-2$ clauses are true, then $(x_1 \lor ... \lor x_n)$ will be true.
		\end{itemize} 
	\item if the clause is $(x_1\lor x_2)$ then create two clauses $(x_1 \lor x_2 \lor y)$ and $(x_1 \lor x_2 \lor \overline{y})$.
	\item if $C_i = (x_1)$ then create four clauses with all possible assignments of $y_1$ and $y_2$.
	\end{itemize}
	By construction $f_{SAT}$ is satisfiable iff $f_{3SAT}$ is satisfiable and the transformation can be done in polynomial time.\\
	\textbf{Problem} (\texttt{independent set})\\
	\underline{Input}: A graph $G$ and an integer $k \leq n(G)$\\
	\underline{Question}: Does $G$ have an independent set (set of non-neighbouring nodes) of order at least $k$?\\
	\textbf{Theorem}\\
	\texttt{Independent Set} $\in$ \textbf{NPC}.

	\section{Optimization problems}
	\subsection{Introduction}
	\textbf{Definition}\\
	An optimization problem $P$ is a 4-tuple $(X, (S_x)_{x\in X}, c, \text{goal})$ where \begin{itemize}
		\item $X$ is a language over $\{0,1\}$that is decidable in polynomial time\item there is a polynomial $p$ s.t. $\forall x \in X$ \begin{itemize}
			\item $S_x$ is a subset if $\{0,1\}^*$
			\item $\left|y\right| \leq p(\left|x\right|)$ for all $y\in S_x$ and 
			\item the language $\{(x,y)| x \in x \text{ and } y\in S_x\}$ is decidable in polynomial time
		\end{itemize}
	\item $c: \{(x,y)|x \in X \text{ and } y \in S_x\} \to \mathbb{Q}$ is computable in polynomial time. $c$ is a measurement of how good the approximated solution is.
	\item goal $\in \{\min, \max\}$
	\end{itemize}
	For $x \in X$ let OPT($x$) = goal$\{c(x,y)|y\in S_x\}$ where $\min \varnothing = \infty$ and $\max \varnothing = -\infty$ and $y \in S_x$ with $c(x,y) =$ OPT($x$) is an \underline{optimal solution} for $x$.\\
	\textbf{Definition}\\
	An algorithm $A$ for an optimization problem $P=(X,S_x,c,\text{goal})$ is a turing machine $\Phi$ with output($\Phi, x$) $\in S_x \; \forall x \in X$ with $S_x \neq \varnothing$. Let $A(x) = c(x,\text{output}(\Phi,x))$. If \[A(x) = \text{OPT}(x) \; \forall x \in X\] with $S_x \neq \varnothing$, then $A$ is an \underline{exact} algorithm. $A$ is an approximation with additive performance guarantee $k$, if \[\left|A(x) - \text{OPT}(x)\right| \leq k \; \forall x \in X \text{ and } S_x \neq \varnothing\] and a $k$-factor approximation algorithm, if \[\frac{1}{k}\cdot \text{OPT}(x) \leq A(x) \leq k\cdot\text{OPT}(x) \; \forall x \in X \text{ and } S_x \neq \varnothing\]
	\subsection{Examples}
	\underline{Comment}: For many problems, a \underline{greedy} approach yields a good approximation factor.\\
	\textbf{Problem} (Minimum Makespan / Job Scheduling with identical machines)\\
	INSTANCE: positive integers $\underbrace{n}_{\text{jobs}},\underbrace{m}_{\text{machines}},\underbrace{a_1,...,a_n}_{\text{required time for job $i$}}$\\

	\noindent GOAL: Determine a function $f: [n] \to [m]$ minimizing \[\max_{j \in [m]} a(f^{-1}(j)) = \max_{j \in [m]} \sum_{i \in [n] \; | \; f(i) = j} a_i\]
	
	\par\noindent\rule{\textwidth}{0.4pt}
	\noindent\underline{Greedy algorithm for job scheduling}:
	\begin{algorithmic}[1]
		\For{$i=1 \text{ to } n$}
			\State $f(i) \gets j, \text{ where $j$ minimizes } \sum_{l \in [i-1] | f(l) = j} a_l$
		\EndFor
		\State return $f$
	\end{algorithmic}
	\par\noindent\rule{\textwidth}{0.4pt}
	We can show, that this is a 2-factor algorithm\\
	Let $l \in [n]$ be the maximum such that $f(l) = k$, where $k \in [m]$ is the machine taking more time. When job $l$ was assigned to $k$ all $m$ machines had completion time at least $t = \sum_{l \in [i-1] | f(l) = k} a()f^{-1}(k))-a_l$ which implies that $a()f^{-1}(k)) = t+a_l \leq 2\cdot\text{OPT}(x)$.
	
	\subsection{Online and Offline Algorithms}
	\textbf{Definition}\\
	An algorithm is called \underline{online} if the input in only partially known before making decisions. I.e. a sequence $a_1,...,a_n$ is given and the decision for $a_{i-1}$ is made before knowing $a_i$. An example is Graham's algorithm for job scheduling.\\
	
	Consider the bin packing problem which is the dual problem of job scheduling. We will give an \underline{offline} algorithm for this optimization problem.\\
	\textbf{Problem} (bin packing)\\
	INSTANCE: $n \in \mathbb{N}$ and $a_1,...,a_n \in \mathbb{Q} \cap [0,1]$\\
	GOAL: Determine the minimal $k \in \mathbb{N}$ and a function \[f: [n] \to [k] \text{ s.t. } a(f^{-1}(j)) = \sum_{i \in [n], f(i) = j} a_i \leq 1 \; \forall j \in [k]\]
	\par\noindent\rule{\textwidth}{0.4pt}
	\underline{First Fit decreasing algorithm}\\
	INPUT $I = (a_1,...,a_n)$
	\begin{algorithmic}[1]
		\State $I \gets I'$ where $I'$ is sorted in a non-increasing way
		\For{$i=1$ to $n$}
			\State $f(j) \gets \min_j ((\sum_{l \in [i-1], f(l) = j} a_l) + a_i \leq 1)$
		\EndFor
		\State return $k,f$
	\end{algorithmic}
	\par\noindent\rule{\textwidth}{0.4pt}
	
	This algorithm puts any given element in the first bin, where the item fits. Therefore, the minimum $j$ is chosen, such that the sum if all items in the $j$-th bin is smaller than one.\\
	\textbf{Theorem}\\
	For a given instance $I$, we have $k \leq \lfloor \frac{3}{2} OPT(I) \rfloor$.\\
	\textbf{Proof}\\
	Let $j = \lceil \frac{2k}{3} \rceil$. Lets call an item $i$ with $a_i > \frac{1}{2}$ big and small otherwise. If bin $m$ has a big item, obviously all bins before $m$ will have a big item as well. Therefore \[OPT(I) \geq j = \lceil \frac{2k}{3} \rceil\] since no two big items can fit in the same bin. Now consider the first bin $l$ with no big items. That is to say, all bins before $l$ have a big item. It follows that bins $l$ to $k$ have only small items and $l$ to $k-1$ pack at least two items. So, $l < \frac{2k+2}{3}$ and then $l-1 \leq 2(k-l)+1$. It follows $OPT(I) \geq a_1 + \dots + a_n > l-1 = \lfloor \frac{2k}{3} + 1\rfloor$.\\
	
	\noindent\textbf{Problem} (3 dimensional matching)\\
	This problem is \textbf{NPC}.\\
	INSTANCE: three disjoint sets $U, V$ and $W$ of equal cardinality and a subset $T \subseteq U\times V \times W$ with $\left|T\right| \geq \left|U\right|$.\\
	GOAL: Decide whether there is a subset $S\subseteq T$ of order $\left|U\right|$ s.t. the elements of $S$ are pairwise different in all components.\\
	\textbf{Theorem}\\
	\texttt{3 Dimensional Matching} is \textbf{NP complete}.
	\textbf{Theorem}\\
	The \texttt{Subset Sum problem} is \textbf{NPC}.

	\section{Greedy Algorithms and Covering Problems}
	\subsection{Feedback Vertex Set}
	INPUT: a graph $G$ and a cost function $c: V(G) \to \mathbb{Q}_{>0}$\\
	GOAL: Determine a set $F$ of vertices of $G$ s.t. $G-F$ is a forest and $c(F) = \sum_{u \in F} c(u)$ is minimum.\\
	
	A set $F$ of vertices of a graph $G$ s.t. $G-F$ is a forest is a \underline{feedback vertex set}. \texttt{Feedback Vertex Set} is a special case of \texttt{Set Cover} where \begin{itemize}
		\item $U$ is the set of all cycles of $G$
		\item $\forall u \in G$ the (mulit-)set $S$ contains all cycles of $G$ that contain $u$
	\end{itemize}
	For a graph $G$ the \underline{cycle space} of $G$ is the subspace of $\mathbb{Z}_2^{E(G)}$ generated by the incidence vectors of cycles of $G$.\\
	\textbf{Claim}\\
	The dimension of the cycle space of $G$ is the \underline{cyclomatic number} \[\mu(G) = m(G) - n(G) + \kappa(G)\] where $\kappa(G)$ is the number of components of $G$.\\
	
	For any $u \in V(G)$ let \[\delta_{\mu_G} (u) = \mu(G) - \mu(G-u)\]
	If $H$ is a subgraph and $u \in V(H)$, then $\delta_{\mu_H}(u) \leq \delta_{\mu_G}(u)$. Therefore, if $F$ is a feedback vertex set, we have $\mu(G) \leq \sum_{u \in F}\delta_{\mu_G}(u)$.\\
	\textbf{Lemma}\\
	If $F$ is a minimal feedback vertex set of some $G$, then \[\sum_{u \in F} \delta_{\mu_G}(u) \leq 2\cdot \mu(G)\]
	\par\noindent\rule{\textwidth}{0.4pt}
	\underline{Algorithm}\\
	INPUT: An instance $(G,c)$ of \texttt{Feedback Vertex Set}\\
	OUTPUT: A feedback vertex set $F$
	\begin{algorithmic}[1]
		\State $G_0 \gets G; c' \gets c; i \gets 0$
		\While{$\mu(G_i) > 0$}
		\State $\varepsilon \gets \min\{\frac{c'(u)}{\delta_{\mu_{G_i}}(u)}: u \in V(G_i)$ and $\delta_{\mu_{G_i}}(u) > 0\}$
		\If{$u \in V(G_i)$}
		\State $c_i(u) \gets \varepsilon \cdot \delta_{\mu_{G_i}}(u)$
		\Else
		\State $c_i(u) \gets 0$
		\EndIf
	\State $c' \gets c' - c_i;\; G_{i+1} \gets G_i[\{u \in V(G_i): c'(u) > 0\}];\; i \gets i+1$
	\EndWhile
	\State $k \gets i;\; c_k \gets c';\; F_k \gets \emptyset$
	\For{$i$ \textbf{from} $k$ \textbf{down to} $1$}
	\State Extend the feedback vertex set $F_i$ of $G_i$ to a feedback vertex set $F_{i-1}$ of $G_{i-1}$ by \indent adding a minimal set of vertices from $V(G_{i-1})\setminus V(G_i)$
	\EndFor
	\State \textbf{return} $F_0$
	\end{algorithmic}
	\par\noindent\rule{\textwidth}{0.4pt}\\
	
	\noindent\textbf{Theorem}\\
	Above algorithm is a polynomial time 2-factor approximation for \texttt{Feedback Vertex Set}.
	\subsection{Maximum Satisfiability}
	\textbf{Definition} (\texttt{Maximum Satisfiability})\\
	INSTANCE: A set $X$ of variables, a set $\mathcal{C}$ of clauses over $X$ and a function $c: \mathcal{C} \to \mathbb{Q}_{>0}$\\
	GOAL: Find a truth assignment $t$ maximizing the total weight  \[\sum_{C \in \mathcal{C}: C \text{ true under } t} c(C)\]
	of satisfied clauses. \texttt{Maximum Satisfiability} is \textbf{NP hard}. $\mathcal{C}$ is a yes-instance of SAT iff the optimum value of \texttt{Maximum Satisfiability} for $\mathcal{C}$ equals the total weight of all clauses.\\
	\textbf{Comment} (Method of conditional expectation)\\
	Let $(X,\mathcal{C},c)$ be an instance of \texttt{Maximum Satisfiability}. if $t$ is a random truth assignment for $X$, we get the expected value by \[\mathbb{E}[T] = \sum_{j \in [m]} c(C_j)\cdot \mathbb{P}(C_j \text{ true under } t)\]
	where $T$ is the random variable that describes the true assignments of $t$. In order to obtain a deterministic approximation algorithm, we apply the \underline{method of conditional expectation}.\\
	For $p_1,...,p_n \in [0,1]$ let \[c(t(x_1),...,t(x_i),r(p_{i+1}),...,r(p_n)) = \mathbb{E}[\text{total weight of satisfied clauses}\]
	where \begin{itemize}
		\item the truth values $t(x_1),...,t(x_i)$ are fixed and 
		\item the remaining $n-i$ truth assignments are chosen randomly. 
	\end{itemize}
	Note that \[c(t(x_1),...,t(x_i),r(p_{i+1}),...,r(p_n)\]
	equals \begin{align*}
		p_i &\cdot c(t(x_1),...,t(x_{i-1}), true,r(p_{i+1}),...,r(p_n))\\ + (1-p_i)&\cdot c(t(x_1),...,t(x_{i-1}),false,r(p_{i+1}),...,r(p_n))\\
		&\leq \max \{c(t(x_1),...,t(x_{i-1}), true,r(p_{i+1}),...,r(p_n)),\\ &c(t(x_1),...,t(x_{i-1}),false,r(p_{i+1}),...,r(p_n))\}
	\end{align*}

	\par\noindent\rule{\textwidth}{0.4pt}
	\underline{Johnson's algorithm}\\
	INPUT: An instance $(x,\mathcal{C}, c)$ of \texttt{maximum satisfiability}\\
	OUTPUT: $t: X \to \{\text{true, false}\}$
	\begin{algorithmic}[1]
		\For{$i=1$ to $n$}
		\If{$c(t(x_1),...,t(x_{i-1}), true,r(p_{i+1}),...,r(p_n)) \geq c(t(x_1),...,t(x_{i-1}),false,r(p_{i+1}),...,r(p_n))$}
		\State $t(x_i) \gets$ true
		\Else
		\State $t(x_i) \gets$ false
		\EndIf
		\EndFor
		\State \textbf{return} $t$
	\end{algorithmic}
	\par\noindent\rule{\textwidth}{0.4pt}\\
	\textbf{Theorem}\\
	For a given instance of \texttt{Maximum Satisfiability} the above algorithm determines a truth assignment $t$ with \[\sum_{C \in \mathcal{C}: C \text{ true under } t} c(C) \geq \frac{1}{2} OPT(X,\mathcal{C},c)\]
	\textbf{Comment} (integer linear program for \texttt{Maximum Satisfiability})\\
	The problem can also be solved using an integer linear program. This yields another algorithm which we analyse in the following. 
	\par\noindent\rule{\textwidth}{0.4pt}
	\underline{Goemans-Williamson's algorithm}\\
	INPUT: An instance $(x,\mathcal{C}, c)$ of \texttt{maximum satisfiability}\\
	OUTPUT: $t: X \to \{\text{true, false}\}$
	\begin{algorithmic}[1]
		\State Let $(t_1,...,t_n) \in [0,1]^n$ be the value from an optimal solution of (P).
		\For{$i=1$ to $n$}
		\If{$c(t(x_1),...,t(x_{i-1}), true,r(t_{i+1}),...,r(t_n)) \geq c(t(x_1),...,t(x_{i-1}),false,r(t_{i+1}),...,r(t_n))$}
		\State $t(x_i) \gets$ true
		\Else
		\State $t(x_i) \gets$ false
		\EndIf
		\EndFor
		\State \textbf{return} $t$
	\end{algorithmic}
	\par\noindent\rule{\textwidth}{0.4pt}\\
	\textbf{Theorem}\\
	For a given instance of \texttt{Maximum Satisfiability} the above algorithm determines a truth assignment $t$ with \[\sum_{C \in \mathcal{C}: C \text{ true under } t} c(C) \geq (1-\frac{1}{\varepsilon})\cdot OPT(X,\mathcal{C},c)\]
	where $1-\frac{1}{\varepsilon} \geq 0.63$.\\
	\textbf{Theorem}\\
	If one combines both algorithm for a given instance, a $\frac{4}{3}$ factor approximation can always be achieved by choosing the better result.
	\subsection{Knapsack}
	\textbf{Definition}\\
	INSTANCE: positive integers $n, c_1,...,c_,.w_1,...,w_n,W$ with $\sum_{j \in [n]} w_j > W$ and $w_j \leq W$.\\
	GOAL: Determine a set $S \subset [n]$ with $\sum_{i \in S} w_i \leq W$ such that $\sum_{i \in S} c_i$ is maximized.\\
	
	This problem can be easily described as an integer linear program: \begin{align*}
		\max &\sum_{i\in [n]} c_i x_i\\
		\text{s.t. } &\sum_{i\in [n]} w_i x_i \leq W\\
		&x_i \in \{0,1\}
	\end{align*}
	Relaxing this by replacing $x_i \in \{0,1\}$ with $x_i \in [0,1]$ yields a simpler linear program $(P)$.\\
	\textbf{Claim}: If \[\frac{c_1}{w_1} \geq ... \geq \frac{c_n}{w_n}\]
	and \[i^* = \min \{i \in [n]: \sum_{j \in [i]} w_j \geq W\}\]
	then we can set \[x_i = \begin{cases}
		1, & 1 \leq i \leq i^*-1\\
		(W-\sum_{j \in [i^*-1]}w_j)/w_{i^*}, & i = i^*\\
		0, & i^*+1 \leq i \leq n
	\end{cases}\]
	and achieve an optimal solution to $(P)$. Note that \[x' = (\underbrace{1,...1}_{i^*-1},0,...,0) \text{ and } x'' = (\underbrace{0,...,0}_{i^*-1},1,0,...,0)\]
	are both feasible solutions for $(P_I)$ with $c^Tx \leq c^T(x'+x'')$, hence \[\max\{c^Tx',\; c^Tx''\} \geq \frac{1}{2} OPT(P) \geq \frac{1}{2} OPT(P_I)\] Therefore returning the better of the two solutions achieves a $2$-factor approximation algorithm. The problem is \textbf{NP}-hard.\\
	\textbf{Theorem}\\
	There is an exact algorithm for \texttt{Knapsack} with running time $\mathcal{O}(size(I)\cdot C)$ for an instance $I$ and an integer $C \geq OPT(I)$. Note that $\sum_{i\in [n]}c_i$ is a possible choice for $C$.\\
	\textbf{Comment}\\
	The following algorithm is an online algorithm that decides for each element based only on \begin{itemize}
		\item the values $c_i$ and $w_i$ of the current $i$-th item
		\item the currently available space
		\item the following global parameters of any instance $I$ \begin{itemize}
			\item[$\rightarrow$] $U = \max\{\frac{c_i}{w_i}: \; i \in [n]\}$
			\item[$\rightarrow$] $L = \min\{\frac{c_i}{w_i}: \; i \in [n]\}$ and 
			\item[$\rightarrow$] $\varepsilon = \max\{\frac{w_i}{W}: \; i \in [n]\}$.
		\end{itemize}
	\end{itemize}

	The decision is based on the function $\psi: [0,1] \to [0,U]$ that has the following attributes: \begin{itemize}
		\item $\psi(0) = \frac{L}{e}$
		\item $\psi(z_L) = L$
		\item $\psi(1) = U$
	\end{itemize}  
	The \underline{competitive ratio} of this algorithm is \[(\frac{Ue}{L})^{2\varepsilon}(1+\ln(\frac{U}{L}))\]
	which is basically the best possible. The competitive ratio is a measurement that describes how good the performance of an algorithm is compared to an optimal offline algorithm. The trivial online algorithm has a competitive ratio of \[\frac{U}{(1-\varepsilon)L}\] because any solution from this algorithm yields a value of at least $L(1-\varepsilon)$.   
\end{document}